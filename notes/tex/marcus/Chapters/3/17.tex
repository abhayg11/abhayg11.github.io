\mtexe{3.17}
\begin{proof}
	Throughout, let $S = \ZZ[\omega]$. \\
	
	First, we have that $f(Q|2)$ is the multiplicative order of 2 mod 23, which is $11$. Then, $f(Q|P)f(P|2) = f(Q|2) = 11$, and $f(P|2) \leq [K:\QQ] = 2$, so we must have $f(P|2) = 1$ and $f(Q|P) = 11$. Then, since $[L:K] = [L:\QQ]/[K:\QQ] = 22/2 = 11$, we get that $PS = Q$, since the sum of $e \cdot f$ over all primes lying over $P$ gives $11$. I.e. $Q = (2R+\theta R)S = 2S + \theta S$ as claimed. \\
	
	We have:
	\[ P^3 = (8,4\theta,2\theta^2,\theta^3) = (8,4\theta,2\theta-12,5\theta+6) \]
	since $\theta^2 = \theta-6$. This equation also gives:
	\[ (\theta-2)^2 = \theta^2-4\theta+4 = -3\theta-2 = -3(\theta-2)-8 \]
	So, $\theta-2$ divides $8$. Then it also divides $4(\theta-2)+8 = 4\theta$, $2(\theta-2)-8 = 2\theta-12$, and $5(\theta-2)+16 = 5\theta+6$. Thus $P^3 \subseteq (\theta-2)$.
	
	Conversely, $\theta-2 = (5\theta+6)-(4\theta)-(8)$, so $\theta-2 \in P^3$. Thus, the two ideals are equal, as claimed.
	
	On the other hand, suppose that $P$ is principal, generated by $\alpha$. Then $P^3 = (\alpha^3) = (\theta-2)$, and so
	\[ 8 = |N_\QQ^K(\theta-2)| = \|P\| = |N_\QQ^K(\alpha^3)| = |N_\QQ^K(\alpha)|^3 \]
	and so $N_\QQ^K(\alpha) = \pm 2$. But if $\alpha = a+b\theta$, then its norm is:
	\[ (a+b\theta)(a+b(1-\theta)) = a^2+ab+b^2(\theta-\theta^2) = a^2+ab+6b^2 = \frac{a^2+11b^2+(a+b)^2}{2} \]
	So, in order for this to be $\pm 2$, we would need $a^2+11b^2+(a+b)^2 = \pm 4$, which must actually be $4$. This forces $b=0$, else it would be too large, and so $2a^2 = 4$, which has no integer solutions. So, $P$ is indeed not principal. \\
	
	From the previous exercise, $3 = d_P \mid d_Qf(Q|P) = 11d_Q$. So, $3 \mid d_Q$ and $Q$ is not principal. \\
	
	Suppose $2 = \alpha\beta$ for some nonunits $\alpha,\beta \in S$. Then, $(\alpha)(\beta) = 2S = QQ'$, where $Q$ is as above and $Q' = (2,1-\theta)$ lies over the other prime $P' = (2,1-\theta$ over 2. But then comparing prime factorizations gives that (WLOG) $Q = (\alpha)$, contradicting the fact that $Q$ is not principal.
\end{proof}
