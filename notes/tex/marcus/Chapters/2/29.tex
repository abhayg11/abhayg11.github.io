\mtexe{2.29} 
\begin{proof} 
    In the first case, let $R = \ZZ[(1+\sqrt{m})/2]$ and $S = \ZZ[(1+\sqrt{n})/2]$, and note these are the rings of integers in each of their fraction fields. Then, $\disc(R) = m$ and $\disc(S) = n$ are coprime, so the ring of integers in $\QQ(\sqrt{m},\sqrt{n})$ is $RS = \ZZ[(1+\sqrt{m})/2, (1+\sqrt{n})/2]$, which has integral basis:
    \[ 1, \frac{1+\sqrt{m}}{2}, \frac{1+\sqrt{n}}{2}, \frac{1+\sqrt{m}+\sqrt{n}+\sqrt{mn}}{4} \]
    and discriminant $(mn)^2$ by Exercise 23(c). \\

    In the second case, let $R = \ZZ[(1+\sqrt{m})/2]$ as before, but let $S = \ZZ[\sqrt{n}]$ to be the corresponding rings of integers again. Then $\disc(R) = m$ and $\disc(S) = 4n$ are again coprime, so the ring of integers in $\QQ(\sqrt{m},\sqrt{n})$ is $RS$ which has integral basis:
    \[ 1, \frac{1+\sqrt{m}}{2}, \sqrt{n}, \frac{1+\sqrt{m}}{2}\sqrt{n} \]
    and discriminant $16m^2n^2$.
\end{proof}
