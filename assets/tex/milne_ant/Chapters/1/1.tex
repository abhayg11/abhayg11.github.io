\mtexe{1.1}
\begin{proof}
	{[Note: this exercise does not require $A$ to be a domain, so long as we are okay with multiplicative sets containing $0$ (in which case the localization is trivial)]} \\
	
	Suppose first that $S$ is a saturated multiplicative subset, and let $T$ denote its complement. Let $x \in T$. I claim its image $x/1 \in S^{-1}A$ is not a unit. Indeed, if it were, then there would be $a/s \in S^{-1}A$ such that $(x/1)(a/s) = 1/1$, i.e. $u(ax-s) = 0$ for some $u \in S$. But then $uax = us \in S$ and by saturation, $x \in S$, contrary to assumption. So, $x/1$ is not a unit in $S^{-1}A$, whence it is contained in a prime ideal of $S^{-1}A$. But this corresponds precisely to a prime ideal $P$ of $A$ containing $x$ that is disjoint from $S$. I.e. we've found a prime ideal of $A$ contained in $T$ that contains $x$. Taking the union over all $x \in T$ gives the result that $T$ is a union of primes, as claimed.
	
	Conversely, suppose that $T = \bigcup_i P_i$ is a union of prime ideals. Let $ab \in S$ for some $a,b \in A$. For each $i$, we have $ab \notin P_i$, so $a \notin P_i$ and $b \notin P_i$. So $a,b$ are both not contained in the union of the $P_i$, i.e. they are not contained in $T$. In other words, they are contained in $S$. \\
	
	Let $\scS$ denote the collection of all saturated multiplicative subsets of $A$ containing $S$. This is a nonempty collection since $A \in \scS$, so we can consider $S' = \bigcap \scS$. I claim $S'$ is a saturated multiplicative subset of $A$. Indeed, if $ab \in S'$, then $ab \in U$ for each $U \in \scS$, whence $a,b \in U$ for each such $U$, so that $a,b \in S'$. Further, it is clear that $1 \in S'$ since $1 \in U$ for each $U \in \scS$, and it is clear that $S'$ is closed under multiplication. This immediately handles the existence and uniqueness questions posed.
	
	Now, we'd like to show $S' = A \setminus \bigcup \frp$. Let $V = A \setminus \bigcup \frp$ for notation. Note that $V$ is a multiplicative set, for if $a,b \in V$, then $a,b \notin \frp$ for each indexed prime, so that $ab \notin \frp$ by primality, and $ab \in V$. Further, it is obvious that the complement of $V$ is a union of primes, so $V$ is saturated. By our definition of $S'$, this gives $S' \subseteq V$. For the reverse, let $x \in V$. If $x/1$ is not a unit in $S^{-1}A$, then there is some prime ideal of $S^{-1}A$ containing $x/1$, which corresponds to a prime ideal of $A$ disjoint from $S$ containing $x$. But this precisely contradicts the definition of $V$. So, $x/1$ is a unit in $S^{-1}A$, which means it has an inverse $a/s$ for some $a \in A$ and $s \in S$. That is, $ax/s = 1/1$, whence $uax = us$ for some $u \in S$, giving $uax \in S$. But then $uax \in S'$ since $S'$ contains $S$, and since $S'$ is saturated, $u,a,x \in S$. In particular, we've shown that $V \subseteq S'$, and so $V = S'$.
	
	Consider now the localization maps $f : A \to S^{-1}A$ and $g : A \to {S'}^{-1}A$. First, note that if $s \in S$, then $s \in S'$, so $g(s) = s/1$ is a unit. By the universal property, we get a unique map $h : S^{-1}A \to {S'}^{-1}A$ with $g = h \circ f$. Similarly, the argument above shows that if $s \in S'$, then $f(s')$ is a unit in $S^{-1}A$ so we get a map $h' : {S'}^{-1}A \to S^{-1}A$ with $f = h' \circ g$. But then $g = (h \circ h') \circ g$ and $f = (h' \circ h) \circ f$, so by the uniqueness part of universal property, we get $h \circ h' = \id$ and $h' \circ h = \id$. I.e. $S^{-1}A \cong {S'}^{-1}A$. Note that in the case that $A$ is a domain, this isomorphism is in fact equality when both localizations are considered as subrings of the field of fractions.
	
	In summary, $S^{-1}A = {S'}^{-1}A$ and $S' = V$ is determined by the primes disjoint from $S$, which are precisely the primes in $A$ that remain prime in $S^{-1}A$, explaining the claimed characterization.
\end{proof}
