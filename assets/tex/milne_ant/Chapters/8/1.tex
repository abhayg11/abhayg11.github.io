\mtexe{8.1}
\begin{proof}
	It suffices to show that there are three irreducible factors of $f(X) = X^3-X^2-2X-8$ in $\QQ_2$. Note that the Newton polygon of $f$ with respect to $|\cdot|_2$ has three distinct slopes, and so we indeed get that $f$ splits completely in $\QQ_2$. As a result, the absolute value is unramified, showing that $2 \nmid \disc(\scO_K/\ZZ)$.
	
	On the other hand, we can directly compute the discriminant of $\ZZ[\alpha]/\ZZ$. Note that the action of $\alpha^2$ on the basis $1,\alpha,\alpha^2$ is via the matrix
	\[ \left(\begin{array}{ccc} 0 & 8 & 8 \\ 0 & 2 & 10 \\ 1 & 1 & 3 \end{array}\right) \]
	and so has trace $0+2+3 = 5$. Directly, $\alpha$ has trace $1$. Hence, we can compute the discriminant mod 2 as:
	\begin{align*}
	\Delta
		&= \det\left(\begin{array} T(1) & T(\alpha) & T(\alpha^2) \\ T(\alpha) & T(\alpha^2) & T(\alpha^3) \\ T(\alpha^2) & T(\alpha^3) & T(\alpha^4)\end{array}\right) \\
		&= \det\left(\begin{array} 3 & 1 & 5 \\ 1 & 5 & T(\alpha^2+2\alpha+8) \\ 5 & T(\alpha^2+2\alpha+8) & T(\alpha^3+2\alpha^2+8\alpha) \end{array}\right) \\
		&= \det\left(\begin{array} 3 & 1 & 5 \\ 1 & 5 & 31 \\ 5 & 31 & 49 \end{array}\right) \\
		&\equiv \det\left(\begin{array} 1 & 1 & 1 \\ 1 & 1 & 1 \\ 1 & 1 & 1 \end{array}\right) \pmod{2} \\
		&\equiv 0 \pmod{2}
	\end{align*}
	since the rows are duplicated. Hence, the discriminant is indeed even.
\end{proof}
