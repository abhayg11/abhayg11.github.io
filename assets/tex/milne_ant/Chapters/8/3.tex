\mtexe{8.3}
\begin{proof}
	Let $f(x) = x^6+2x^5+3x^4+4x^3+5x^2+6x+7$. First, consider $f$ modulo 3. We can compute $f'(x) \equiv x^4+x \equiv x(x+1)^3 \pmod{3}$, and we can also compute $f(0) = 7$ and $f(-1) = 4$, neither of which is zero modulo $3$. Hence, $f$ will split into distinct irreducibles modulo 3, giving the form of a permutation in the Galois group.
	
	To figure out the splitting, note that if $f$ is reducible, it has an irreducible factor of degree 1, 2, or 3. If $g$ is such an irreducible factor, then adjoining a root $\alpha$ of $g$ gives a field extension $\FF_{3^r}$, in which case $\alpha$ satisfies both $f$ and $x^{3^r}-x$, and so the minimal polynomial of $\alpha$ divides both of these. So, we should consider their greatest common divisor. For $r=1$, we have already seen that $0,2$ are not roots, so noticing that
	\[ f(1) = 28 \not\equiv 0 \pmod{3} \]
	shows that $\gcd(f,x^3-x) = 1$ in $\FF_3$. For $r=2$, we have
	\begin{align*}
	\gcd(f,x^9-x)
		&= \gcd(f,(x^3-x)(x^6+x^4+x^2+1)) \\
		&= \gcd(x^6-x^5+x^3-x^2+1,x^6+x^4+x^2+1) \\
		&= \gcd(x^5+x^4-x^3-x^2,x^6+x^4+x^2+1) \\
		&= \gcd(x^2(x-1)(x+1)^2,x^6+x^4+x^2+1) \\
		&= 1
	\end{align*}
	where the last line comes from evaluating the right-hand polynomial at $0,1,-1$. Finally, for $r=3$, notice first that
	\[ x^8+x+1 = (x^6-x^5+x^3-x^2+1)(x^2+x+1) \]
	So, $x^8 \equiv -x-1 \pmod{f}$. Hence,
	\[ x^{26} - 1 \equiv (x^8)^3x^2 - 1 \equiv (-x-1)^3x^2-1 \equiv -x^5-x^2-1 \pmod{f} \]
	This helps with computing the $\gcd$:
	\begin{align*}
	\gcd(f,x^{27}-x)
		&= \gcd(f,x^{26}-1) \\
		&= \gcd(f,x^5+x+1) \\
		&= \gcd(f,(x-1)^2(x^3-x^2+1)) \\
		&= \gcd(x^6-x^5+x^3-x^2+1,x^3-x^2+1) \\
		&= \gcd(x^6-x^5,x^3-x^2+1) \\
		&= \gcd(x^5(x-1),x^3-x^2+1) \\
		&= 1
	\end{align*}
	Hence, we must conclude that $f$ is irreducible in $\FF_3$. But then it is also irreducible in $\ZZ$, and so in $\QQ$. Further, we can conclude by Dedekind that the Galois group contains a $6$-cycle. \\
	
	Now, we will consider $f$ modulo $13$.
\end{proof}

f(x) = x^6+2x^5+3x^4+4x^3+5x^2+6x+7


mod 2: div by (x+1)^2			no conc
mod 3: x^6-x^5+x^3-x^2+1		
mod 5: 
mod 7: x(x-1)^2(x^3+4x^2+3x+6)	no conc

x^6+2x^5+3x^4+4x^3+5x^2+6x+7




