\mtexe{8.3}
\begin{proof}
	Let $f(x) = x^6+2x^5+3x^4+4x^3+5x^2+6x+7$. First, consider $f$ modulo 3. We can compute $f'(x) \equiv x^4+x \equiv x(x+1)^3 \pmod{3}$, and we can also compute $f(0) = 7$ and $f(-1) = 4$, neither of which is zero modulo $3$. Hence, $f$ will split into distinct irreducibles modulo 3, giving the form of a permutation in the Galois group.
	
	To figure out the splitting, note that if $f$ is reducible, it has an irreducible factor of degree 1, 2, or 3. If $g$ is such an irreducible factor, then adjoining a root $\alpha$ of $g$ gives a field extension $\FF_{3^r}$, in which case $\alpha$ satisfies both $f$ and $x^{3^r}-x$, and so the minimal polynomial of $\alpha$ divides both of these. So, we should consider their greatest common divisor. For $r=1$, we have already seen that $0,2$ are not roots, so noticing that
	\[ f(1) = 28 \not\equiv 0 \pmod{3} \]
	shows that $\gcd(f,x^3-x) = 1$ in $\FF_3$. For $r=2$, we have
	\begin{align*}
	\gcd(f,x^9-x)
		&= \gcd(f,(x^3-x)(x^6+x^4+x^2+1)) \\
		&= \gcd(x^6-x^5+x^3-x^2+1,x^6+x^4+x^2+1) \\
		&= \gcd(x^5+x^4-x^3-x^2,x^6+x^4+x^2+1) \\
		&= \gcd(x^2(x-1)(x+1)^2,x^6+x^4+x^2+1) \\
		&= 1
	\end{align*}
	where the last line comes from evaluating the right-hand polynomial at $0,1,-1$. Finally, for $r=3$, notice first that
	\[ x^8+x+1 = (x^6-x^5+x^3-x^2+1)(x^2+x+1) \]
	So, $x^8 \equiv -x-1 \pmod{f}$. Hence,
	\[ x^{26} - 1 \equiv (x^8)^3x^2 - 1 \equiv (-x-1)^3x^2-1 \equiv -x^5-x^2-1 \pmod{f} \]
	This helps with computing the $\gcd$:
	\begin{align*}
	\gcd(f,x^{27}-x)
		&= \gcd(f,x^{26}-1) \\
		&= \gcd(f,x^5+x+1) \\
		&= \gcd(f,(x-1)^2(x^3-x^2+1)) \\
		&= \gcd(x^6-x^5+x^3-x^2+1,x^3-x^2+1) \\
		&= \gcd(x^6-x^5,x^3-x^2+1) \\
		&= \gcd(x^5(x-1),x^3-x^2+1) \\
		&= 1
	\end{align*}
	Hence, we must conclude that $f$ is irreducible in $\FF_3$. But then it is also irreducible in $\ZZ$, and so in $\QQ$. Further, we can conclude by Dedekind that the Galois group contains a $6$-cycle. \\
	
	Now, we will consider $f$ modulo $13$. <INCLUDE THE WORK TO SHOW THAT IT SPLITS AS 5,1>. \\
	
	Finally, consider $f$ module $19$ <THIS GIVES 4,1,1 I THINK>. \\
	
	Overall, we have shown the galois group $G$ of $f$ to be a transitive subgroup of $S_6$ containing a 4-cycle, a 5-cycle, and a 6-cycle. Hence, $|G| \geq 60$, and explicitly enumerating subgroups of $S_6$ requires that $G$ is isomorphic to one of $A_5,S_5,A_6,S_6$. Note that a 6-cycle is an odd permutation, so it cannot be $A_6$. There are two distinct conjugacy classes of $S_5$ in $S_6$. The usual one, obtained as the stabilizer of a point, is not transitive, so cannot be $G$, nor can its contained copy of $A_5$. The other one, however, arises from the outer isomorphism of $S_6$ applied to this subgroup; this is transitive, so we must consider it (another way to see that it is transitive is that the injection $S_5 \hookrightarrow S_6$ arises from the conjugation action of $S_5$ on its sylow-5 subgroups, which is transitive). Nonetheless, note that the only element of order 4 in $S_5$ must be a 4-cycle, so even though this embedding doesn't preserve cycle type in general, the 4-cycle in $G \subseteq S_6$ would correspond to a 4-cycle in $S_5$ under this embedding, which is again an odd permutation.
	
	So, we conclude that either $G = S_6$ or $G = S_5 \subseteq S_6$ via the exceptional embedding. Investigating other small primes suggests that we do not have all elements of $S_6$, and so we would guess (without full proof) that $G = S_5$. \\
	
	Finally, we do this computation with a computer. In Sage, this can be accomplished directly via:
	\begin{verbatim}
sage: (x^6 + 2*x^5 + 3*x^4 + 4*x^3 + 5*x^2 + 6*x + 7).galois_group(pari_group=True)
PARI group [120, -1, 14, "L(6):2 = PGL(2,5) = S_5(6)"] of degree 6
	\end{verbatim}
	Here, we see that the group order is $120$, and sage (PARI) has identified the group as being isomorphic to $PGL_2(\FF_5) \cong S_5$. This confirms our guess above.
	
	For more detail, we can also compute factorizations themselves. Running the following gives us the degrees of the factors after reducing modulo $p$ for various primes, discarding the factorizations with repeated roots (i.e. those that ramify):
	\begin{verbatim}
sage: for p in primes(100):
....:     u = list(f.change_ring(GF(p)).factor())
....:     ram = False
....:     for t in u:
....:         if t[1] > 1:
....:             ram = True
....:             break
....:     if ram:
....:         continue
....:     print(p,[t[0].degree() for t in u])
....:     
3 [6]
5 [3, 3]
11 [6]
13 [1, 5]
17 [1, 5]
19 [1, 1, 4]
23 [1, 1, 4]
29 [1, 5]
31 [1, 1, 4]
37 [3, 3]
41 [1, 5]
43 [1, 1, 4]
47 [6]
53 [3, 3]
59 [1, 1, 4]
61 [1, 1, 2, 2]
67 [6]
71 [6]
73 [1, 1, 2, 2]
79 [2, 2, 2]
83 [1, 1, 4]
89 [1, 5]
97 [3, 3]
	\end{verbatim}
	Notice that the ratios that appear for the various cycle types are: $([6], 5/23)$, $([3,3], 4/23)$, $([1,5], 5/23)$, $([1,1,4], 6/23)$, $([1,1,2,2], 2/23)$, $([2,2,2], 1/23)$.
\end{proof}

f(x) = x^6+2x^5+3x^4+4x^3+5x^2+6x+7


mod 2: div by (x+1)^2			no conc
mod 3: x^6-x^5+x^3-x^2+1		
mod 5: 
mod 7: x(x-1)^2(x^3+4x^2+3x+6)	no conc

x^6+2x^5+3x^4+4x^3+5x^2+6x+7




