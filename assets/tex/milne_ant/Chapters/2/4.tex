\mtexe{2.4}
\begin{proof}
	First, it is clear that $\fra \neq (2)$ since $1+\sqrt{-3} \in \fra$, but $1+\sqrt{-3} \notin (2)$ since $\frac{1+\sqrt{-3}}{2} \notin \ZZ[\sqrt{-3}]$ since $\{1,\sqrt{-3}\}$ is a basis for $\QQ(\sqrt{-3})$ over $\QQ$. Directly, we have:
	\[ \fra^2 = (4,2+2\sqrt{-3},4) = 2(2,1+\sqrt{-3}) = 2\fra \]
	This shows that we do not have uniqueness of factorization of ideals into primes. Indeed, if we did, then writing $(2) = \frp_1 \cdots \frp_r$ and $\fra = \frq_1 \cdots \frq_m$ gives the distinct factorizations
	\[ \frp_1 \cdots \frp_r \cdot \frq_1 \cdots \frq_m = \frq_1^2 \cdots \frq_m^2 \]
	for $\fra^2 = 2\fra$; if these were not distinct then we would conclude $\fra = (2)$.
\end{proof}
