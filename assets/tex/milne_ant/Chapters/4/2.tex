\mtexe{4.2}
\begin{proof}
	First, write the factorization of $\frD B$, and note that one of the factors is $\frP^{e(\frD/\frP)}$. Then, similarly, write the factorization of $\frP A$ and note that one of the factors is $\frp^{e(\frP/\frp)}$. Substitute the latter expression into the former to get the factorization of $\frD A$, which includes the factor
	\[ \left(\frp^{e(\frP/\frp)}\right)^{e(\frD/\frP)} = \frp^{e(\frD/\frP)e(\frP/\frp)} \]
	By uniqueness of factorization, this exponent must be exactly $e(\frD/\frp)$ as claimed.
	
	The statement about inertial degrees is even more direct, since degrees multiply in towers of field extensions. Namely:
	\[ f(\frD/\frP)f(\frP/\frp) = [C/\frD:B/\frP][B/\frP:A/\frp] = [C/\frD:A/\frp] = f(\frD/\frp) \]
\end{proof}
