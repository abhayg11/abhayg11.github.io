\mtexe{4.4}
\begin{proof}
	Note that for $K = \QQ(\sqrt{-23})$, the ring of integers is $\scO_K = \ZZ[\alpha]$ for $\alpha = (1+\sqrt{-23})/2$ with minimal polynomial $f(x) = x^2-x+6$ and discriminant $\Delta = -23$. The Minkowski bound is:
	\[ \frac{2!}{2^2}\left(\frac{4}{\pi}\right)^1|-23|^{1/2} = (2/\pi)\sqrt{23} < 4 \]
	So, each ideal class has an integral representative of norm 1, 2, or 3. The only integral ideal of norm 1 is $\scO_K$ itself, representing the trivial ideal class. Any ideal of norm 2 divides $(2)$, so we start by considering the factorization of $2$, which requires factoring $f$ mod $2$:
	\[ f(x) = x^2-x+6 \equiv x(x-1) \pmod{2} \]
	So $(2) = (2,\alpha)(2,\alpha-1)$ and each of these has norm 2. Similarly analyzing mod 3 gives:
	\[ f(x) = x^2-x+6 \equiv x(x-1) \pmod{3} \]
	so that $(3) = (3,\alpha)(3,\alpha-1)$. Since $N(\alpha) = N(\alpha-1) = 6$, we get $(\alpha) = (2,\alpha)(3,\alpha)$ and $(\alpha-1) = (2,\alpha-1)(3,\alpha-1)$. So we get
	\[ (2,\alpha) \sim (3,\alpha)^{-1} \sim (3,\alpha-1) \sim (2,\alpha-1)^{-1} \]
	where $\sim$ denotes equivalence of ideal classes. It remains to show that $(1),(2,\alpha),(2,\alpha-1)$ are pairwise distinct. To see that neither $(2,\alpha)$ nor $(2,\alpha-1)$ are principal, it suffices to show that no element of $\scO_K$ has norm two. But
	\[ N(a+b\alpha) = (a+b\alpha)(a+b(1-\alpha)) = a^2+ab+6b^2 = \frac{1}{4}(2a+b)^2 + \frac{23}{4}b^2 \]
	If $b \neq 0$, then $N(a+b\alpha) \geq 23/4 > 2$, so it cannot be 2. So $b=0$, and $N(a) = a^2 \neq 2$.
	
	Finally, it remains to show that $(2,\alpha) \not\sim (2,\alpha-1)$, for which it suffices to show that $(2,\alpha)^2$ is not principal. If it were principal, then again we'd find $a+b\alpha$ with norm 4. Again, if $b \neq 0$, then the norm is too big, so we must have $b=0$ and $a = \pm 2$. So, it suffices to show that $(2,\alpha)^2 \neq (2)$. But finally, comparing factorizations means that it suffices to show that $(2,\alpha) \neq (2,\alpha-1)$. This is true, since if they were equal, that ideal would contain $(\alpha)-(\alpha-1) = 1$, and so wouldn't be proper, whereas we know that it is prime. So, the class number is exactly 3. \\
	
	Now, we use the same approach for $K = \QQ(\sqrt{-47})$, $\scO_K = \ZZ[\alpha]$ for $\alpha = (1+\sqrt{-47})/2$ of minimal polynomial $f(x) = x^2-x+12$ and discriminant $\Delta = -47$. The Minkowski bound is:
	\[ \frac{2!}{2^2}\left(\frac{4}{\pi}\right)^1|-47|^{1/2} = (2/\pi)\sqrt{47} < 5 \]
	Now we seek integral ideals of norm 1, 2, 3, or 4. Again the only ideal of norm 1 is $(1)$.
	
	As above, when considered either modulo 2 or 3, we get that $f(x)$ splits as $x(x-1)$. So $(2) = (2,\alpha)(2,\alpha-1)$ and $(3) = (3,\alpha)(3,\alpha-1)$. Again considering norms gives
	\[ (\alpha) = (2,\alpha)^2(3,\alpha) \text{ and } (\alpha-1) = (2,\alpha-1)^2(3,\alpha-1) \]
	So, the ideal classes represented by all of the above primes are in the cyclic subgroup generated by the class of $(2,\alpha)$. Now, if $I$ is an integral ideal of norm 4, then each of its prime factors divides 2, so must be one of $(2,\alpha),(2,\alpha-1)$. Comparing norms shows that $I$ is a product of exactly two such factors, so $I$ is also a power of $(2,\alpha)$ in the ideal class group.
	
	So, it suffices to find the order of $(2,\alpha)$ in the ideal class group. First, note:
	\begin{align*}
	(2,\alpha)^2 &= (4,2\alpha,\alpha^2) = (4,2\alpha,\alpha-12) = (4,\alpha) \\
	(2,\alpha)^3 &= (8,4\alpha,2\alpha,\alpha^2) = (8,2\alpha,\alpha-12) = (8,\alpha-4) \\
	(2,\alpha)^5 &= (32,8\alpha,4\alpha-16,\alpha^2-4\alpha) = (32,8\alpha,4\alpha-16,-3\alpha-12) = (\alpha+4)
	\end{align*}
	where the final equality follows from:
	\[ 32 = (\alpha+4)(5-\alpha) \text{ and } 8\alpha = 8(\alpha+4)-32 \text{ and } 4\alpha-16 = 4(\alpha+4)-32 \text{ and } -3\alpha-12 = -3(\alpha+4) \]
	So, the order of $(2,\alpha)$ divides 5. It finally remains to show that it isn't itself principal, for which it suffices to show that there is no element of norm 2. But
	\[ N(a+b\alpha) = (a+b\alpha)(a+b(1-\alpha)) = a^2+ab+12b^2 = \frac{1}{4}(2a+b)^2 + \frac{47}{4}b^2 \]
	For this to equal 2, we must have $b=0$, lest it be too big, but then $a^2 = 2$, which has no integer solutions. So we're done and the ideal class group is cyclic of order 5.
\end{proof}
