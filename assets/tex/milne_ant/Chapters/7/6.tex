\mtexe{7.6}
\begin{proof}
	Without loss of generality, after reordering, we may assume $p_i$ is odd for all $i$ except possibly $i=1$. Let $K_i = \QQ(\alpha_1,\ldots,\alpha_i)$, so that $K_0 = \QQ$ and $K_m = K$. It suffices to show that $K_i \neq K_{i-1}$ for all $i$, since each extension is of degree at most 2. For $i = 1$, this is obvious, since $\sqrt{p_1} \in \QQ(\sqrt{p_1}) \setminus \QQ$. Now, fix $i \geq 2$. Let $n = 4p_1 \cdots p_{i-1}$, let $\zeta_n$ be a primitive $n$th root of unity, and let $L = \QQ(\zeta_n)$. Then for $j < i$ we have $\zeta_{p_j} \in L$, so $\QQ(\zeta_{p_j}) \subseteq L$, and so $\sqrt{\pm p_j} \in L$, since $\QQ(\sqrt{\pm p_j})$ is the unique quadratic subextension of $\QQ(\zeta_{p_j})$. Further, we have $\sqrt{-1} = \zeta_n^{n/4} \in L$, so we also have $\sqrt{p_j} \in L$ for all $j < i$. Hence, $K_{i-1} \subseteq L$. On the other hand, we cannot have $\sqrt{p_i} \in L$, for if it were, then $p_i$ would ramify in $\scO_L$. But for cyclotomic extensions, we know that this would imply $p_i \mid n$, which cannot be since $p_i$ is an odd prime distinct from each of $p_1,\ldots,p_{i-1}$. So, $K_i \not\subseteq L$, and so it cannot be that $K_i = K_{i-1}$, as desired.
	
	Now, let $\gamma = \sum_{i=1}^m \sqrt{p_i}$. We will show by induction on $m$ that $K = \QQ(\gamma)$. For $m=1$, this is again obvious. Suppose this is known for all smaller $m$, and for contradiction, suppose $K \neq \QQ(\gamma)$. Then $\QQ(\gamma)$ is fixed by some element of the Galois group of $K/\QQ$. But we can identify these directly; namely, for any $S \subseteq \{1,\ldots,m\}$, we have the automorphism $\sigma_S$ given by
	\[ \sigma_S(\alpha_i) = \begin{cases} \alpha_i & \text{if }i \notin S \\ -\alpha_i & \text{if }i \in S \end{cases} \]
	This indeed gives the $2^m$ automorphisms we expect. But it is obvious that $\gamma - \sigma_S(\gamma) \neq 0$ for any nonempty $S$ since it is positive as twice the sum of square roots.
	
	Hence the minimal polynomial $f$ of $\gamma$ is irreducible of degree $2^m$. Fix a prime $p$, and note that if a field $K$ contains all quadratic extensions of $\QQ_p$, then it contains each square root $\sqrt{p_i}$, so contains all the roots of $f$ by taking signed sums, and so we can deduce an upper bound on the degrees of irreducible factors of $f$ by the degree $[K:\QQ_p]$. But $\QQ_p$ has characteristic zero (in particular, not 2), and so all quadratic extensions are of the form $\QQ_p(\sqrt{a})$ for some $a$ in $\QQ_p$, and two such extensions are the same whenever $a,a'$ differ by a square factor. In other words, we are seeking coset representatives for $\QQ_p^\times/(\QQ_p^\times)^2$.
	
	We determined for $p=2$ in the previous problem that there are eight cosets, generated by at most three cyclic direct summands. So, there are numbers $a,b,c$ such that each quadratic extension is given by adjoining $\sqrt{a^ib^jc^k}$. So, we can take $K = \QQ_2(\sqrt{a},\sqrt{b},\sqrt{c})$, which has degree at most 8, showing the claim. For $p \neq 2$, we can use a similar argument. Here, $\QQ_p^\times = \left<p\right> \oplus \ZZ_p^\times$, so $\QQ_p^\times/(\QQ_p^\times)^2 \cong \{1,p\} \oplus \ZZ_p^\times/(\ZZ_p^\times)^2$. But $\ZZ_p^\times = \mu_p(1+p\ZZ_p)$, where $\mu_p$ denotes the set of Hensel lifts of each root of $X^{p-1}-1$, corresponding to the nonzero elements of $\FF_p$. So, $(\ZZ_p^\times)^2 = \mu_p^2(1+p\ZZ_p)$ is an index 2 subgroup of $\ZZ_p^\times$, and so overall the quotient is of order 4, completing the argument as before.
\end{proof}
