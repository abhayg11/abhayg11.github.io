\mtexe{4.1.4}
\begin{proof}
	This is very direct:
	\begin{align*}
	\left<\overline{x_1},\overline{y}\right>\left<\overline{x_2},\overline{y}\right>
		&= \chi(x_1yx_1^{-1}y^{-1})\chi(x_2yx_2^{-1}y^{-1}) \\
		&= \chi((x_1yx_1^{-1}y^{-1})(x_2yx_2^{-1}y^{-1})) \\
		&= \chi(x_1(x_2yx_2^{-1}y^{-1})yx_1^{-1}y^{-1}) \\
		&= \chi(x_1x_2yx_2^{-1}x_1^{-1}y^{-1}) \\
		&= \left<\overline{x_1}\overline{x_2},\overline{y}\right>
	\end{align*}
	where we've used that the right commutator is in the center. The second follows from the first and the fourth, as:
	\[ \left<\overline{x},\overline{y_1}\overline{y_2}\right> = \left<\overline{y_1}\overline{y_2},\overline{x}\right>^{-1} = \left(\left<\overline{y_1},\overline{x}\right>\left<\overline{y_2},\overline{x}\right>\right)^{-1} = \left<\overline{x},\overline{y_1}\right>\left<\overline{x},\overline{y_2}\right> \]
	as desired. For the third,
	\[ \left<\overline{x},\overline{x}\right> = \chi(xxx^{-1}x^{-1}) = \chi(1) = 1 \]
	and for the fourth,
	\[ \left<\overline{y},\overline{x}\right>^{-1} = \chi(yxy^{-1}x^{-1})^{-1} = \chi(xyx^{-1}y^{-1}) = \left<\overline{x},\overline{y}\right> \]
	completing the computation.
\end{proof}
