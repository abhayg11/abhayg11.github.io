\mtexe{2.31} 
\begin{proof} 
    Let $\alpha = (\sqrt{3}+\sqrt{7})/2$. Then,
    \[ 4\alpha^2 = 10+2\sqrt{21} \]
    so that:
    \[ 84 = (4\alpha^2-10)^2 = 16\alpha^4 - 80\alpha^2 + 100 \]
    i.e.
    \[ \alpha^4 - 5\alpha^2 + 1 = 0 \]
    so that $\alpha$ is an algebraic integer. But $\alpha$ isn't in $RS$ where $R = \ZZ[\sqrt{3}]$ and $S = \ZZ[\sqrt{7}]$ are the rings of integers of their respective fraction fields, despite $\alpha$ being in the compositum field.
\end{proof}
