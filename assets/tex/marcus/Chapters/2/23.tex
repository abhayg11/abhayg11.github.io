\mtexe{2.23}
\begin{proof}
	The generalized definition for the discriminant should be the following: given an extension of number fields $L/K$ of degree $n$, consider the $K$-embeddings $\sigma_1,\ldots,\sigma_n$ of $L$ into $\CC$. Define
	\[ \disc^L_K(\alpha_1,\ldots,\alpha_n) = \det((\sigma_i(\alpha_j))_{ij})^2 \]
	for any $n$-tuple of elements $\alpha_1,\ldots,\alpha_n \in L$.
	
	Then Theorem 6 generalizes immediately:
	\[ \disc^L_K(\alpha_1,\ldots,\alpha_n) = \det((T^L_K(\alpha_i\alpha_j))_{ij}) \]
	The proof is the same as in the text, since it makes no explicit reference to $\QQ$ or $\ZZ$. \\
	
	Similarly, the corollary generalizes: $\disc^L_K(\alpha_1,\ldots,\alpha_n) \in K$, and if each $\alpha_i$ is an algebraic integer, then $\disc^L_K(\alpha_1,\ldots,\alpha_n) \in \scO_K$, which follows from the above since $T^L_K$ has image in $K$, and maps algebraic integers to elements of $\scO_K$. \\
	
	Theorem 7 now reads: $\disc^L_K(\alpha_1,\ldots,\alpha_n) = 0$ iff $\alpha_1,\ldots,\alpha_n$ are linearly dependent over $K$. Again, the proof goes through exactly as before. \\
	
	Finally, theorem 8: Supposing $L = K[\alpha]$ and $\alpha_1,\ldots,\alpha_n$ are the images of $\alpha$ in $\CC$ under the embeddings, we have
	\[ \disc^L_K(1,\ldots,\alpha^{n-1}) = \prod_{1 \leq r < s \leq n} (\alpha_r-\alpha_s)^2 = (-1)^{n(n-1)/2}N^L_K(f'(\alpha)) \]
	where $f$ is the minimal polynomial of $\alpha$ over $K$. The proof again requires little to no modification. \\
	
	Let the notation be as in the suggestion. Then both $A$ and $B$ have a natural partitioning into block matrices of size $m \times m$, allowing us to compute that the product $AB$ also consists of an $n \times n$ grid of $m \times m$ submatrices, with the $(i,j)^\text{th}$ submatrix having $(h,k)^\text{th}$ entry
	\[ \sigma_i(\tau_h(\beta_k))\sigma_i(\alpha_j) = \sigma_i(\alpha_j\tau_h(\beta_k)) = \sigma_i(\tau_h(\alpha_j\beta_k)) \]
	In other words, as $i,j,k,h$ vary, we get precisely that $AB$ has entries corresponding to the application of the $mn$ embeddings of $M$ into $\CC$, applied to each of the basis elements for $M/K$. So, the squared determinant gives exactly the discriminant:
	\[ \det(AB)^2 = \disc_K^M(\alpha_1\beta_1,\ldots,\alpha_n\beta_m) \]
	We now study $\det(A)^2$ and $\det(B)^2$ independently. The latter is more straightforward: $B$ is obtained by taking a matrix and replacing each entry with an $m \times m$ copy of the identity matrix times that entry, so overall, the determinant is raised to the $m^\text{th}$ power. I.e. $\det(B)^2 = \disc_K^L(\alpha_1,\ldots,\alpha_n)^m$.
	
	Since $A$ is block-diagonal, we can compute its determinant as the product of the determinants of the blocks. Consider the $n \times n$ matrix $A'$ whose $(h,k)$-th entry is $\tau_h(\beta_k)$. Similarly, we have that the entries are given by the $L$-embeddings of $M$ into $\CC$, applied to a basis of $M/L$, so $\det(A')^2 = \disc_L^M(\beta_1,\ldots,\beta_n)$. But then the $i$th block of $A$ is $\sigma_i(A')$ (applied entrywise), so
	\[ \det(A)^2 = \prod_{i=1}^m \det(\sigma_i(A'))^2 = \prod_{i=1}^m \sigma_i(\det(A')^2) = N_K^L(\disc_L^M(\beta_1,\ldots,\beta_n)) \]
	The result now follows from the equality $\det(AB) = \det(A)\det(B)$.
	
	Let $\alpha_1,\ldots,\alpha_n$ and $\beta_1,\ldots,\beta_m$ be integral bases for $R$ and $S$, respectively. Under the stated hypotheses, $T = RS$, so $\alpha_i\beta_j$ is an integral basis for $T$. From the above,
	\begin{align*}
	(\disc T)
		&= \disc_\QQ^{KL}(\alpha_i\beta_j) \\
		&= \disc_\QQ^K(\alpha_1,\ldots,\alpha_n)^{[KL:L]}N_\QQ^K(\disc_K^{KL}(\beta_1,\ldots,\beta_m)) \\
		&= (\disc R)^{[L:\QQ]}(\disc S)^{\disc R)^{[K:\QQ]}
	\end{align*}
	since $\beta_1,\ldots,\beta_m$ is a basis for $L/K$, and so $\disc_K^{KL}(\beta_1,\ldots,\beta_m) = \disc_\QQ^L(\beta_1,\ldots,\beta_m)$ and the norm simply multiplies this integer by itself $[K:\QQ]$ times.
\end{proof}
