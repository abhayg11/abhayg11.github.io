\mtexe{1.6}
\begin{proof}
    Suppose the claim is not true. Then, choose $\alpha \in \ZZ[i]$ such that $\alpha$ is a nonzero, nonunit Gaussian integer such that $\alpha$ is not a product of irreducibles, with $N(\alpha)$ as small as possible (by well-ordering of the nonnegative integers). We cannot have $N(\alpha) = 0$ or $N(\alpha) = 1$, since it would be zero or a unit in these cases, respectively. Further, $\alpha$ is not itself irreducible, else it would be a single product. So, there are nonunits $\beta,\gamma$ with $\alpha = \beta\gamma$. Then $N(\alpha) = N(\beta)N(\gamma)$, and so $N(\beta),N(\gamma) < N(\alpha)$, since neither factor is equal to 1. By minimality, we can write $\beta,\gamma$ as the product of irreducibles, but then $\alpha$ is the product of all of these irreducibles together.
\end{proof}