\mtexe{1.27}
\begin{proof}
    We may assume, without loss of generality, that $0 \leq k < p$. If $k=0$, then we get that $p$ divides $y - y\omega^2$, but problem 22 implies that $p \mid y$, contrary to assumption. If $2 \leq k \leq p-2$, we get that $p$ divides
    \[ x + y\omega - x\omega^k - y\omega^{k-1} \]
    and again, problem 22 implies that $p$ divides each coefficient, including, say, $x$. Finally, if $k=p-1$, we have that $p$ divides
    \[ x+y\omega-x\omega^{p-1}-y\omega^{p-2} \]
    and so it also divides (multiplying by $\omega^2$):
    \[ x\omega^2+y\omega^3-x\omega-y \]
    and so it divides $y$. The only remaining possibility is that $k=1$, as claimed.
\end{proof}