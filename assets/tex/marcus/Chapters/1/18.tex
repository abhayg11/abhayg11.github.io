\mtexe{1.18}
\begin{proof}
    Let $P$ be the set of prime divisors of $x+y\omega$. Then, each $\pi \in P$ also divides $z^p$ and so $z$. Thus, the unique factorization of $z$ gives an integer $n$ with $\pi^n \mid z$ and $\pi^{n+1} \nmid z$. Since $\pi$ does not divide any of the other terms in the product, we have that $\pi^{np}$ divides $x+y\omega$ but no higher power. Hence, each prime divisor of $x+y\omega$ has an exponent which is a multiple of $p$. I.e. $x+y\omega = u\alpha^p$, where $\alpha$ is the product of $\pi^n$ over all $\pi \in P$ (and corresponding exponents $n$), and $u$ is the unit in the factorization.
\end{proof}