\mtexe{3.36}
\begin{proof}
	It suffices to show that $\alpha_1,\ldots,\alpha_n$ are independent over $K$. Let $c_1,\ldots,c_n \in K$ be such that
	\[ c_1\alpha_1 + \cdots + c_n\alpha_n = 0 \]
	Clearing denominators, we may assume that $c_i \in R$ for each $i$. Then, reducing mod $PS$, we get a similar dependence relation. The images of the $\alpha_i$ are independent in $S/PS$ (over $R/P$), so we conclude that $c_i \in P$ for each $i$. So, we have $(c_1,\ldots,c_n) \subseteq P$ as ideals in $R$. If the former is nonzero, then by a lemma in the text, we can find $\gamma \in K$ such that $\gamma(c_1,\ldots,c_n)$ is still an ideal of $R$ not contained in $P$. But the exact same argument applied to $\gamma\sum_i c_i\alpha_i = 0$ gives $\gamma c_i \in P$ for each $i$, contrary to construction. So, we must have $(c_1,\ldots,c_n) = 0$, so each $c_i$ is zero, showing that $\{\alpha_1,\ldots,\alpha_n\}$ is independent over $K$. \\
	
	It is immediate that $PA \subseteq A$ and $PA \subseteq PS$, so we are looking for the reverse containment. Suppose $x \in PS \cap A$. We can write
	\[ x = \sum_j r_j\alpha_j \]
	for some $r_j \in R$. Since $x \in PS$, this gives a relation modulo $PS$; since the images of the $\alpha_j$ form a basis there over $R/P$, we conclude that $r_j \in P$ for each $j$. But then we've shown that $x \in PA$ as desired. \\
	
	As suggested, suppose that $I \subseteq P$. Then $IS \subseteq PS$, and for each $r \in I$, $rS \subseteq A$ by assumption. Since these generate $IS$ as an $R$-module, we conclude that $IS \subseteq A$ as well, so by the above, $IS \subseteq PA$. But then $P^{-1}IS \subseteq A$, and so for any $r \in P^{-1}I$, $rS \subseteq A$, showing $r \in I$. I.e. $P^{-1}I \subseteq I$, which contradicts factorization. \\
	
	Note that $IS \subseteq A \subseteq S$ by assumption. So, $\diff(IS) \subseteq \diff(A) \subseteq \diff(S)$, and rewriting as divisibility and factoring out the fractional ideal $IS$ gives the result. To conclude the statement about powers of $Q$, it would suffice to show that $Q \nmid IS$ since then $\diff(A)$ must have exponent on $Q$ between that of $\diff(S)$ and zero greater. If $Q \mid IS$, then $IS \subseteq Q$, and $I = IS \cap R \subseteq Q \cap R = P$, contrary to assumption, and establishing the rest of the claim. \\
	
	In this, the totally ramified case, we may choose $\alpha_i = \pi^{i-1}$ as our basis. Indeed, the construction in exercise 3.20 constructs this basis precisely by choosing $B_1 = \{1\}$ as our singleton basis for $S/Q$ over $R/P$, since these are isomorphic fields, and choosing $\alpha_{1,j} = \pi^{j-1}$. Then, the exponent on $Q$ in $\diff(S)$ is the same as the exponent on $Q$ in $\diff(A) = \diff(R[\pi]) = f'(\pi)S$, where the last equality comes from the previous exercise. 
\end{proof}
