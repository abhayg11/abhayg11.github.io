\mtexe{3.31}
\begin{proof}
	Note that a fractional ideal is an $R$-submodule of $K$. The product as defined is then the submodule of $K$ generated by all products of elements in each submodule, which gives a definition independent of representatives. \\
	
	Clearly $II^{-1} \subseteq R$ since each product $xy$ with $x \in I$ and $y \in I^{-1}$ satisfies $xy \in R$. So, $II^{-1}$ is an ideal of $R$. Assume, for sake of contradiction, that $II^{-1}$ is a proper ideal. Then there is some $\gamma \in K-R$ such that $\gamma II^{-1} \subseteq R$. Then $\gamma I^{-1} \subseteq I^{-1}$. By considering the determinant of the matrix describing the multiplication by $\gamma$ map, we conclude that $\gamma$ is integral over $R$, but since $R$ is normal, this implies that $\gamma \in R$, furnishing our contradiction. Hence, the fractional ideals of $R$ form a group, with $R$ as the identity element. \\
	
	Let $(x/y) I$ be a fractional ideal for $I$ an ideal of $R$ and $x,y \in R$. Then we can factorize the ideals $xR,yR,I$ into products of primes, and the factorization of the fractional ideal is clear.
	
	Let $G = \{ \alpha R \mid \alpha \in K^\times \}$ denote the free abelian group of principal fractional ideals. We have a group homomorphism $K^\times \to G$ given by $\alpha \mapsto \alpha R$. An element $\alpha$ is in the kernel iff $\alpha R = R$ iff $\alpha \in R^\times$. Hence, we get that $G \cong K^\times/R^\times$ is the multiplicative group of $K$ mod units in $R$. \\
	
	Note that if $R$ is a PID then the set of nonzero principal ideals is the set of all nonzero ideals, which is a free abelian semigroup since $R$ is Dedekind. Conversely, suppose $R$ is a Dedekind domain with the nonzero principal ideals forming a free abelian semigroup. Let $B$ be a basis.

	We would like to show that $R$ is a PID, for which it suffices to show that $R$ is a UFD since $R$ is Dedekind. Since $R$ is Noetherian, the only thing to check is the uniqueness. That is, suppose
	\[ x_1 \cdots x_n = y_1 \cdots y_m \]
	for irreducibles $x_i,y_i \in R$. I claim that each $x_iR,y_iR \in B$. Indeed, for $t=x_i$ or $t=y_i$, we can, by assumption, write
	\[ tR = (z_1R) \cdots (z_rR) \]
	for some principal ideals $z_iR \in B$, not necessarily distinct. WLOG, we may assume that each $z_iR$ is a proper ideal, else we omit it. But then $t = uz_1 \cdots z_r$ for a unit $u$, and by irreducibility, we must have $r=1$ and $t = uz_1$, whence $tR = z_1R \in B$. Thus, the expressions
	\[ \prod_i x_iR = \prod_j y_jR \]
	are the unique factorization of these principal ideals relative to the basis $B$. Hence, they must agree, i.e. $n=m$ and $x_iR = y_iR$ after rearranging. That is, $x_i = u_iy_i$ for units $u_i$ for each $i$, which completes the proof that $R$ is a UFD. \\
	
	By assumption, each fractional ideal in $G$ is of the form $\alpha I$ for $\alpha \in K^\times$ and $I$ a nonzero ideal. Let $C$ denote that ideal class group of $R$. Consider the map $\varphi : G \to C$ defined by:
	\[ \alpha I \mapsto [I] \]
	First, we need to show that this is well-defined. Indeed, the same fractional ideal can have different representatives. Suppose $(x/y)I = (x'/y')I'$ for ideals $I,I'$ and $x,y,x',y' \in R$. But then $xy'I = x'yI'$, and so $I \sim I'$, i.e. $[I]=[I']$. It is also clear that $\varphi$ is a group homomorphism, for
	\[ \varphi((\alpha I)(\beta J)) = \varphi(\alpha\beta IJ) = [IJ] = [I][J] = \varphi(\alpha I)\varphi(\beta J) \]
	To finish the proof, we will show $\varphi$ is surjective with kernel $H$. For the first, note that each ideal class $[I]$ has a representative $I$, which is the image of $1I$. For the latter, note that a principal fractional ideal is of the form $\alpha R$, which maps to $[R]$, the identity. Conversely, if $\varphi(\alpha I) = [R]$, then $[I] = [R]$, so $xI = yR$, whence $I = (y/x)R$ is principal. So, $C \cong G/H$ with isomorphism $\varphi$. \\
	
	Finally, note that $R$ is Noetherian so if $\alpha I$ is an arbitrary fractional ideal, then $I = \sum_{i=1}^r x_1R$ for some $x_i \in R$, whence $\alpha I = \sum_{i=1}^r \alpha x_iR$ is finitely generated as an $R$-module. Conversely, suppose that $M \subseteq K$ is a nonzero finitely generated $R$-module. That is,
	\[ M = \sum_{i=1}^r \alpha_iR \]
	for some $\alpha_i \in R$. Let $y$ be the product of the denominators of the $\alpha_i$, so that $y\alpha_i \in R$ for each $i$. Then,
	\[ M = \sum_{i=1}^r \frac{y\alpha_i}{y}R = \frac{1}{y}\left(\sum_{i=1}^r y\alpha_iR\right) \]
	and the parenthesized expression is an $R$-submodule of $R$, i.e. an ideal. So $M = (1/y)I$ is a fractional ideal.
\end{proof}
