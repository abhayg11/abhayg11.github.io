\mtexe{3.35}
\begin{proof}
	This is exactly what we get by applying $\sigma_i$ to both sides of $f(x) = (x-\alpha)g(x)$ since $f \in K[x]$ is fixed by each $\sigma_i$. \\
	
	Since we are in characteristic zero, the roots of $f$ are distinct, so if $i \neq j$ then $\alpha_i \neq \alpha_j$. Since $\alpha_j$ is a root of $f$ but not of $x-\alpha_i$, we thus conclude $\alpha_j$ is a root of $g_i$. This gives $g_i(\alpha_j) = 0$ for $i \neq j$ as claimed.
	
	For the case $i=j$, let us differentiate the above conclusion:
	\[ f'(x) = g_i(x) + (x-\alpha_i)g_i'(x) \]
	Thus, evaluating at $\alpha_i$ gives $g_i(\alpha_i) = f'(\alpha_i)$, also as claimed. \\
	
	Directly, the $(i,j)-$th entry of $NM$ is
	\begin{align*}
	\sum_{k=1}^n \sigma_i(\gamma_{k-1}/f'(\alpha))\alpha_j^{k-1}
		&= \frac{1}{\sigma_i(f'(\alpha))}\sum_{k=0}^{n-1} \sigma_i(\gamma_k)\alpha_j^k \\
		&= \frac{g_i(\alpha_j)}{f'(\alpha_i)}
		&= \frac{f'(\alpha_j)\delta_{ij}}{f'(\alpha_i)} \\
		&= \delta_{ij}
	\end{align*}
	So, $NM = I$ as claimed. As square matrices with entries in a field, this implies that $MN = I$ as well, which has $(i,j)-$th entry:
	\[ \sum_{k=1}^n \alpha_k^{i-1}\sigma_k(\gamma_{j-1}/f'(\alpha)) = \sum_{k=1}^n \sigma_k(\alpha^{i-1}\gamma_{j-1}/f'(\alpha)) = T(\alpha^{i-1}\gamma_{j-1}/f'(\alpha)) \]
	The fact that this is the kronecker delta tells us that $\{1,\alpha,\ldots,\alpha^{n-1}\}$ and $\{\gamma_0/f'(\alpha),\ldots,\gamma_{n-1}/f'(\alpha)\}$ are dual bases as claimed. \\
	
	Note that $(x-\alpha)g(x) = f(x)$ and $f \in K[x]$. If $\alpha \in S$, then in fact $f \in R[x]$. Expanding this multiplication explicitly and investigating the coefficient of $x^i$ gives that
	\[ \gamma_{i-1}-\alpha\gamma_i \in R \]
	for $i=0,\ldots,n-1$, where we define $\gamma_{-1} = 0$ for convenience. The coefficient of $x^n$ also gives $\gamma_{n-1} = 1$. But now, induction makes it clear that $\gamma_i \in R[\alpha]$ for all $i$; it is certainly true for $i=n$ and the above recurrence allows us to pass from $i$ to $i-1$.
	
	Conversely, the same recurrence shows that $M = R\gamma_0 + \cdots + R\gamma_{n-1}$ contains $R[\alpha]$. Indeed, we will show inductively that it contains $\alpha^i$ for $i \geq 0$. First, since again $\gamma_{n-1} = 1$, we have that $\alpha^0 = 1 \in M$. If $\alpha^i \in M$, then write
	\[ \alpha^i = c_0\gamma_0 + \cdots + c_{n-1}\gamma_{n-1} \]
	for $c_0,\ldots,c_{n-1} \in R$. Multiplying by $\alpha$ gives
	\[ \alpha^{i+1} = c_0\alpha\gamma_0 + \cdots + c_{n-1}\alpha\gamma_{n-1} = c_0\gamma_{-1} + \cdots + c_{n-1}\gamma_{n-2} = c_1\gamma_0 + \cdots + c_{n-1}\gamma_{n-2} \in M \]
	as desired. \\
	
	From the previous exercise, we showed that $A^\ast = B$, where $A$ has a given basis over $R$ and $B$ has the dual basis. In the case $A = R[\alpha]$, we can use the basis $1,\ldots,\alpha^{n-1}$ and the dual basis $\gamma_0/f'(\alpha),\ldots,\gamma_{n-1}/f'(\alpha)$ to conclude
	\[ R[\alpha]^\ast = R\gamma_0/f'(\alpha) + \cdots + R\gamma_{n-1}/f'(\alpha) = (f'(\alpha))^{-1}(R\gamma_0 + \cdots + R\gamma_{n-1}) = (f'(\alpha))^{-1}R[\alpha] \]
	as desired. \\
	
	Then, by the definition of the different,
	\[ \diff(R[\alpha]) = (R[\alpha]^\ast)^{-1} = ((f'(\alpha))^{-1}R[\alpha])^{-1} = f'(\alpha)S \]
	For the final equality, note that if $x \in ((f'(\alpha))^{-1}R[\alpha])^{-1}$, then $xf'(\alpha){-1}R[\alpha] \subseteq S$ and in particular, since $1 \in R[\alpha]$, $xf'(\alpha)^{-1} \in S$, so $x \in f'(\alpha)S$. Conversely, if $x \in f'(\alpha)S$, then
	\[ xf'(\alpha)^{-1}R[\alpha] \subseteq (f'(\alpha)S)(f'(\alpha)^{-1}R[\alpha]) \subseteq R[\alpha]S \subseteq S \]
	so $x \in (f'(\alpha)^{-1}R[\alpha])^{-1}$ by definition. \\
	
	Note that $R[\alpha] \subseteq S$, and both $\cdot^\ast$ and $\cdot^{-1}$ are inclusion-reversing, so $\diff(R[\alpha]) \subseteq \diff(S)$. But $f'(\alpha) = f'(\alpha) \cdot 1 \in f'(\alpha)S = \diff(R[\alpha])$, so this suffices for the claim.
\end{proof}
