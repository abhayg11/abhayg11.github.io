
\documentclass{article}

\usepackage{amsthm}
\usepackage{libertine}
\usepackage[margin=.7in]{geometry}
\usepackage{amsmath,amssymb}
\usepackage{multicol}
\usepackage[shortlabels]{enumitem}
\setlist[enumerate,1]{label=(\alph*)}
\usepackage{cancel}
\usepackage{graphicx}
\usepackage{listings}
\usepackage{tikz}
\usepackage{mathrsfs}
\usepackage{hyperref}
\hypersetup{
    colorlinks=true,
    linkcolor=blue,
    filecolor=blue,
    urlcolor=blue,
    }
\urlstyle{same}
\usepackage{tikz-cd} 

% Theorem environments
\newtheorem*{definition}{Definition}
\newtheorem*{thm}{Theorem}
\newtheorem*{lem}{Lemma}
\newtheorem*{exercise}{Exercise}
\newenvironment{exe}[1]{\begin{exercise}[#1\label{#1}]}{\end{exercise}}
\newcommand{\mtexe}[1]{\noindent\textbf{Exercise} (#1).}

% Script letter shorthands
\newcommand{\scA}{\mathscr{A}}
\newcommand{\scC}{\mathscr{C}}
\newcommand{\scF}{\mathscr{F}}
\newcommand{\scG}{\mathscr{G}}
\newcommand{\scH}{\mathscr{H}}
\newcommand{\scI}{\mathscr{I}}
\newcommand{\scJ}{\mathscr{J}}
\newcommand{\scL}{\mathscr{L}}
\newcommand{\scM}{\mathscr{M}}
\newcommand{\scN}{\mathscr{N}}
\newcommand{\scO}{\mathscr{O}}
\newcommand{\scS}{\mathscr{S}}
\newcommand{\scZ}{\mathscr{Z}}

% Caligraphic sytle
\newcommand{\mcO}{\mathcal{O}}
\newcommand{\mcP}{\mathcal{P}}
\newcommand{\mcF}{\mathcal{F}}

% Ideals and other fraktures
\newcommand{\fra}{\mathfrak{a}}
\newcommand{\frd}{\mathfrak{d}}
\newcommand{\frm}{\mathfrak{m}}
\newcommand{\frp}{\mathfrak{p}}
\newcommand{\frq}{\mathfrak{q}}
\newcommand{\frD}{\mathfrak{D}}
\newcommand{\frP}{\mathfrak{P}}

% Blackboard style
\renewcommand{\AA}{\mathbb{A}}
\newcommand{\CC}{\mathbb{C}}
\newcommand{\FF}{\mathbb{F}}
\newcommand{\NN}{\mathbb{N}}
\newcommand{\PP}{\mathbb{P}}
\newcommand{\QQ}{\mathbb{Q}}
\newcommand{\RR}{\mathbb{R}}
\newcommand{\ZZ}{\mathbb{Z}}

% Functions
\newcommand{\Cl}{\operatorname{Cl}}
\newcommand{\Frac}{\operatorname{Frac}}
\newcommand{\Hom}{\operatorname{Hom}}
\newcommand{\Mor}{\operatorname{Mor}}
\newcommand{\Pic}{\operatorname{Pic}}
\newcommand{\Proj}{\operatorname{Proj}}
\newcommand{\Span}{\operatorname{span}}
\newcommand{\Spec}{\operatorname{Spec}}
\newcommand{\Tr}{\operatorname{Tr}}
\newcommand{\codim}{\operatorname{codim}}
\newcommand{\diff}{\operatorname{diff}}
\newcommand{\disc}{\operatorname{disc}}
\newcommand{\height}{\operatorname{ht}}
\newcommand{\id}{\operatorname{id}}
\newcommand{\im}{\operatorname{im}}
\newcommand{\lcm}{\operatorname{lcm}}
\newcommand{\lgnd}[2]{\left(\frac{#1}{#2}\right)}
\newcommand{\scHom}{\operatorname{\mathscr{H}om}}


% Other
\newcommand*{\twomat}[4]{\left(\begin{array}{cc} #1 & #2 \\ #3 & #4 \end{array}\right)}
\newcommand*{\prob}[1]{\hyperref[#1]{(#1)}}



\begin{document}
\pagestyle{plain}
\thispagestyle{empty}

\noindent
\begin{tabular*}{\textwidth}{l @{\extracolsep{\fill}} r @{\extracolsep{6pt}} l}
\textbf{PAWS - Abelian Varieties over Finite Fields} & \textbf{Name:} & Abhay Goel \\
\textbf{Problem Set 1} & \textbf{Date:} & \today \\
\end{tabular*} \\
\rule[2ex]{\textwidth}{2pt}

\mtexe{Problem 1}
\begin{enumerate}
	\item When I last checked, LMFDB has \href{https://www.lmfdb.org/Variety/Abelian/Fq/?g=1}{6184} elliptic curves defined over finite fields.
	
	\item Of these, \href{https://www.lmfdb.org/Variety/Abelian/Fq/?g=1&p_rank_deficit=1}{184} (i.e. 2.98%) were supersingular.
	
	\item There are \href{https://www.lmfdb.org/Variety/Abelian/Fq/?g=1&curve_point_count=1&showcol=curve_count}{3}, over $\FF_2$, $\FF_3$, and $\FF_4$.
	
	\item Up to isomorphism, there are 5 elliptic curves over $\FF_2$. One set of representative equations is:
	\begin{align*}
		y^2 + y &= x^3 + x + 1 \\
		y^2 + xy + y &= x^3 + 1 \\
		y^2 + y &= x^3 \\
		y^2 + xy &= x^3 + 1 \\
		y^2 + y &= x^3 + x 
	\end{align*}
	\begin{enumerate}
		\item The first, third, and fifth are supersingular, while the others are not.
		\item They have 1 through 5 $\FF_2$ points, respectively.
		\item The first (with 1 rational point) has Frobenius endomorphism with characteristic polynomial $x^2-2x+2$.
		\item The isogeny class of the elliptic curve with 1 rational point over $\FF_2$ has $L$-polynomial $1-2x+2x^2$, which is the reciprocal polynomial of the characteristic above.
	\end{enumerate}
\end{enumerate}

\mtexe{Problem 2}
\begin{enumerate}
	\item We recall some basics. In the chart $z \neq 0$, we get $E$ in the form $y^2 = x^3+17$. So, given points $(a,b)$ and $(c,d)$ on this curve, we can explicitly write the line between them as solutions of $y = b+\lambda(x-a)$ for
	\[ \lambda = \frac{d-b}{c-a} \]
	If the third point of intersection with $E$ is $(u,v)$, then
	\[ x^3 + 17 = (b+\lambda(x-a))^2 \]
	The coefficient of $x^2$ is clearly $-\lambda^2$, so by the factorization $(x-a)(x-c)(x-u)$, we get
	\[ a+c+u = \lambda^2 \]
	i.e. $u = \lambda^2-a-c$. Then $v$ is immediate and the point $(u,-v)$ is the sum of these points on $E$. \\
	
	For the case of $(a,b) = (-2,3)$ and $(c,d) = (4,9)$, we have that $\lambda = 1$, so $u = -1$ and $v = 4$. Hence $P+Q = [-1:-4:1]$. Repeating this with $(a,b)=(-2,3)$ and $(c,d)=(-1,-4)$, we have $\lambda = -7$, so $u = 52$ and $v = -375$. So, $2P+Q = [52:375:1]$.
	
	We can also consider multiples of an individual point, e.g. $2P$. For this, $\lambda = \frac{3a^2}{2b} = 2$ is the slope of the tangent line, so $u = 8$ and $v = 23$. So $2P = [8:-23:1]$.
	
	Finally, most easily, we can compute $-P = [-2:-3:1]$ and $-Q=[4:-9:1]$ since these points form lines with $P,Q$, respectively, that pass through $O = [0:1:0]$.
	
	\item Done.
	
	\item Done.
\end{enumerate}

\mtexe{Problem 3}
\begin{enumerate}
	\item Note that clearly $[0:1:0]$ is 2-torsion (as the identity). The only other points of $E$ that are 2-torsion are points in the chart $z \neq 0$ with $y = 0$. I.e. they are of the form $[x:0:1]$, so they are solutions to $x^3-x=0$. These are $-1,0,1$ in $\overline{\QQ}$. So, the $2$-torsion points are $[0:1:0],[-1:0:1],[0:0:1],[1:0:1]$.
	
	\item Again, the identity is $3$-torsion. Otherwise, let $P = [x:y:1]$, and note that $3P = 0$ iff $2P=-P$. We have $-P = [x:-y:1]$, so it remains to compute $2P=[u:-v:1]$. From the notes, we have $u = \lambda^2-2x$ and $v=y+\lambda(u-x)$ for
	\[ \lambda = \frac{3x^2-1}{2y} \]
	So, when $2P=-P$, we get $u=x$, which immediately implies $y=v$ as desired, and also requires $\lambda^2 = 3x$. I.e.
	\[ 9x^4 - 6x^2 + 1 = (3x^2-1)^2 = 12xy^2 = 12x(x^3-x) = 12x^4 - 12x^2 \]
	and so
	\[ 0 = x^4 - 2x^2 - 1/3 = (x^2 - 1)^2 - 4/3 \]
	i.e. $x^2 - 1 = \pm 2/\sqrt{3}$, and so $x = \pm \sqrt{1 \pm 2/\sqrt{3}}$ has four solutions over $\overline{\QQ}$. Each of these gives two points on $E$, so we have a total of nine 3-torsion points on $E$ (including the identity).
	
	\item I'm not sure what this means; $\QQ$ is not a local field. Note that the discriminant is 64, which is zero in $\FF_2$, so indeed this is singular over $\FF_2$ and has bad reduction.
	
	\item As the discriminant is not zero in $\FF_3$, it indeed defines an elliptic curve there. For the computation of 3-torsion, much of the above still works, except of course division by 3. So, we seek solutions to:
	\[ 0 = 3x^4 - 6x^2 - 1 = -1 \]
	which has no solutions. So, there are no 3-torsion points over $\overline{\FF_3}$. Hence, $\overline{E}$ must be supersingular.
	
	\item TBD
\end{enumerate}

\mtexe{Problem 4}
\begin{enumerate}
	\item Directly:
	\[ \Delta(E^{(p)}) = -(4B^p)^2(-B^{2p})-8(2B^p)^3 = 16B^{4p}-16B^{3p} = (16B^4-16B^3)^p \]
	since $16 \in \FF_p$, and
	\[ \Delta(E) = 16(4A^3+27B^2) \]
	So I don't think the claim is true. If the $p$-Frobenius twist of $E$ was defined by the Weierstrass equation
	\[ y^2z = x^3+A^pxz^2+B^pz^3 \]
	then we would have
	\[ \Delta(E^{(p)}) = 16(4(A^p)^3+27(B^p)^2) = \Delta(E)^p \]
	as claimed. In this case, we would also have:
	\[ j(E^{(p)}) = -1728\frac{(4A^p)^3}{\Delta(E^{(p)})} = -1728\left(\frac{(4A)^3}{\Delta(E)}\right)^p = j(E)^p \]
	again using that $4,1728 \in \FF_p$. Hence $E^{(p)}$ is defined by a nonsingular Weierstrass equation and so is an elliptic curve.
	
	\item To see that $\phi_p$ is a map of abelian varieties, it suffices to note that it is clearly a map of varieties, given by polynomial equations and mapping solutions of the first equation into solutions of the second, and it preserves the identity, since $\phi_p[0:1:0] = [0^p:1^p:0^p] = [0:1:0]$. To see that it is an isogeny, we wish to show that it surjects with finite kernel.
	
	The surjectivity is clear, since $\overline{\FF_q}$ is perfect, so every element is a $p^\text{th}$ power. Further, if a point $[a:b:c]$ is in the kernel, then $[a^p:b^p:c^p] = [0:1:0]$, so $a=c=0$ and $[a:b:c] = [0:b:0] = [0:1:0]$. So, the kernel is in fact trivial.
\end{enumerate}
\end{document}





















