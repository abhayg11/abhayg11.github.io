\mtexe{2.2.1} 
\begin{proof}
	Following the explicit construction given, the sheafification of the constant presheaf is the sheaf $\scA^\dagger$ where $\scA^\dagger(U)$ is the set of functions $f : U \to \bigsqcup_{x \in U} \scA_x$ given locally by sections. Clearly $\scA_x = A$, and the fact that $f$ is locally given by sections means that for each $x \in U$, there is some open $V$ and $a \in \scA(V) = A$ with $f(x) = a_x = a$ for each $x \in V$. Another way to state this is that $f : U \to A$ is continuous when $A$ is given the discrete topology. I.e. $\scA^\dagger$ is the sheaf of continuous functions from subsets of $X$ to $A$ with the discrete topology, i.e. locally constant functions $X \to A$. Finally, we can interpret this as $\scA^\dagger(U)$ being the direct product of copies of $A$ in bijection with the connected components of $U$. \\
	
	If every open subset of $X$ is connected, then this description shows that $\scA = \scA^\dagger$, so $\scA$ is a sheaf. Conversely, if there is some $U \subseteq X$ that is the disjoint union of open subsets $V,W$ of $X$, then $\scA$ is not a sheaf. Indeed, $A$ is nontrivial, so we can choose two distinct elements $a,b \in A$. Then $a \in \scA(V)$ and $b \in \scA(W)$, and these sections agree on overlaps since $V \cap W = \emptyset$, but there is no $x \in \scA(U)$ with $x|_V = x = a$ and $x|_W = x = b$. So, $\scA$ does not satisfy gluing.
\end{proof}
