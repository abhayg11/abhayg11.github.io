\mtexe{2.1.5} 
\begin{proof}
	Note that injectivity at the beginning and surjectivity at the end are obvious. So, we need to see exactness in the middle. Further, any multiple of $P_1(T_1)$ is clearly in the kernel, so the image is contained in the kernel. Finally, let $f \in k[T_1,\ldots,T_n]$ be in the kernel. Write it as a polynomial in $T_2,\ldots,T_n$ with coefficients in $k[T_1]$, as:
	\[ f = \sum_i a(T_1)T^i \]
	where $i$ runs over multi-indices and $T^i$ denotes $T_2^{i_2} \cdots T_n^{i_n}$. Then, the image in $k_1[T_2,\ldots,T_n]$ is:
	\[ \sum_i \overline{a(T_1)}T^i \]
	which must be zero. I.e. each coefficient must be zero, so $a(T_1) \in P_1(T_1)k[T_1]$. Thus, factoring out $P_1(T_1)$ from each summand shows that $f$ is a multiple of $P_1(T_1)$. \\
	
	Now, we induct on $n$. Indeed, $P_1(T_1) \subseteq \frm$, so $\frm$ corresponds to a maximal ideal $\frm_1$ of $k_1[T_2,\ldots,T_n]$. Let $p_2(T_2)$ generate $\frm_1 \cap k_1[T_2]$. Then, lift this back to a polynomial $P_2(T_1,T_2) \in k[T_1,T_2]$. Notice that if $f \in k[T_1,T_2] \cap \frm$, then the image is in $\frm_1 \cap k_1[T_2]$, so $f$ is contained in $(P_1,P_2)$. Let $k_2 = k_1[T_2]/(p_2)$ and continue in this way to get the desired sequence.
\end{proof}
