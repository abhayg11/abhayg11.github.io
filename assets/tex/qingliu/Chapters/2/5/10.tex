\mtexe{2.5.10} 
\begin{proof}
	For $X = \AA^1_k$, the result is immediate: each $Y_n$ is closed, so it is of the form $V(I)$ for $I$ an ideal of $k[T]$. But this is a PID, so $I = (f)$. If $f = 0$, then $V(I) = X$, which doesn't have strictly smaller dimension than $X$, so we must have $f \neq 0$. But then $f$ only has finitely many prime factors, so $V(I)$ is finite, i.e. each $Y_n$ is finite. So, $\bigcup_n Y_n$ is a countable set, while $X$ is uncountable, since each $(T-a)$ is a maximal ideal of $k[T]$ for each $a \in k$ and there are uncountably many of these.
	
	This now implies the case for $X = \AA^m_k$ by induction on $m$. Indeed, first again write $Y_n$ as $V(I)$ for some ideal $I$. Since $k[T_1,\ldots,T_m]$ is Noetherian, $I = (f_1,\ldots,f_r)$ is finitely generated. Considering each $f_i$ as a polynomial in $T_m$ with coefficients in $k[T_1,\ldots,T_{m-1}]$, they each have finitely many roots. In other words, overall, $I$ is contained in only finitely many ideals of the form $(T_m - a)$ for $a \in k$. So, the set $S = \{(Y_n,a) \mid (T_m-a) \in Y_n \}$ is countable, as a countable union of finite sets. On the other hand, for $a \in k$, consider the map $\varphi_a : k[T_1,\ldots,T_m] \to k[T_1,\ldots,T_{m-1}]$ that maps $T_m$ to $a$ and is the identity on the rest. This induces a map $f_a : \AA^{m-1}_k \to \AA^m_k$ and if $\bigcup_n Y_n = \AA^m_k$, then $\bigcup_n f_a^{-1}(Y_n) = \AA^{m-1}_k$. By induction, this implies $f_a^{-1}(Y_n) = \AA^{m-1}_k$ for some $n$. But then $(T_m-a) = \ker(\varphi_a) = \varphi^{-1}(0) = f_a(0) \in Y_n$. This shows that the set $S$ above is uncountable, since for each $a \in k$ it contains $(Y_n,a)$ for some $n$. \\
	
	Now, if $X = \Spec B$ is an affine variety, so $B$ is a finitely generated $k$-algebra, then we can write $B$ as a finite $A = k[T_1,\ldots,T_m]$-module for a subring $A \subseteq B$. But then the induced map $f : X \to \Spec A$ is closed. Indeed, if $V(I)$ is a closed subset of $X$, then $f(V(I)) = V(A \cap I)$. One containment is obvious: if $Q \subseteq B$ is a prime containing $I$, then $Q \cap A$ is a prime of $A$ containing $A \cap I$. For the other containment, note that if $P \in \Spec A$ contains $A \cap I$, then the composition $A \to B \to B/I$ is integral, so satisfies lying-over, giving a prime $Q$ of $B$ containing $I$ with $Q \cap A = P$.
	
	But now, if $X = \bigcup_n Y_n$, then $\Spec A = \bigcup_n f(Y_n)$ (again using lying-over for surjectivity of $f$), and so by the previous work $f(Y_n) = \Spec A$ for some $n$. But $\dim(f(Y_n)) \leq \dim(Y_n) < \dim(X) = \dim(\Spec A)$ (first inequality from a prior exercise) so this cannot be.
	
	Finally, if $X$ is now an arbitrary algebraic variety and $X = \bigcup_n Y_n$, then $U = \bigcup_n (U \cap Y_n)$ for any affine $U \subseteq X$. But $\dim(U \cap Y_n) \leq \dim(Y_n) < \dim(X) = \dim(U)$, so this cannot be either.
\end{proof}
