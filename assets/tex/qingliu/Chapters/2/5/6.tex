\mtexe{2.5.6} 
\begin{proof}
	Let $\frp$ be a prime ideal of $A$ containing $I$, and so in particular containing $x$. Suppose we have a chain of primes $\frp_0 \subsetneq \cdots \subsetneq \frp_n = \frp$. By a lemma in the section, there is another chain of primes $\frq_1 \subsetneq \cdots \frq_n = \frp$ with $x \in \frq_1$. This gives a chain of prime ideals in $A/xA$ ending with $\frp/xA$, and so we conclude $n-1 \leq \height(\frp/xA)$, i.e. $n \leq \height(\frp/xA)+1$. Taking the supremum over all $n$, i.e. over all chains of primes, we conclude that $\height(\frp) \leq \height(\frp/xA)+1$. Taking the minimum over all $\frp$ containing $I$ gives us the conclusion we want: $\height(I) \leq \height(I/xA)+1$. In particular, we're using here the fact that as $\frp$ varies over all primes of $A$ containing $I$, $\frp/xA$ enumerates all primes of $A/xA$ containing $I/xA$, since the ideals of $A/xA$ correspond to ideals of $A$ containing $x$, primes are preserved under this correspondence, and any prime containing $I$ automatically contains $x$.
	
	On the other hand, the inequality $\height(I/xA) \leq \height(I)$ is obvious, since any chain of primes in $A/xA$ lifts to one in $A$. Thus, if the previous inequality is strict, we must have $\height(I/xA) = \height(I)$. Choose a prime $\frp$ containing $I$ with $\height(I/xA) = \height(\frp/xA)$ and a chain of primes in $A/xA$ ending with $\frp/xA$. This lifts to a chain of primes in $A$ containing $x$, and the first prime in this chain must be minimal in $A$, else we would get $\height(I) > \height(I/xA)$. So, this first prime is a minimal prime of $A$ containing $x$. \\
	
	Suppose now that $I$ is contained in the union of minimal primes of $A$. Since $A$ is noetherian, there are only finitely many of these, and by prime avoidance, it must then be contained in one of them. But then $\height(I) = 0$. Conversely, suppose $I$ is not contained in the union of minimal primes. Then choose $x \in I$ that is in no minimal prime and note that by the above argument, $\height(I) = \height(I/xA)+1 \geq 1$. \\
	
	Finally, we show the last claim by induction on $r$. If $r=0$, then the statement is trivial by taking $J = 0$. Suppose now $r \geq 1$. By the above, $\height(I) = r \geq 1$ means that we can choose $x \in I$ not contained in any minimal prime of $A$. Then $\height(I/xA) = \height(I)-1 = r-1$, so by induction, we can choose an ideal $J'$ generated by $r-1$ elements with $\height(J') = r-1$ and $\height((I/xA)/J') = 0$. Lifting these generators gives an ideal $J$ of $A$ with $J/xA = J'$. Then $J+xA$ is generated by $r$ elements and since $x$ is still not contained in any minimal prime of $A$, we conclude that $\height(J+xA) = \height((J+xA)/xA)+1 = \height(J')+1 = r$. Finally, note that since $(I/xA)/J'$ has height zero, there is a minimal prime of $(A/xA)/J' \cong A/(J+xA)$ containing $I$. But this isomorphism shows that there is a minimal prime of $A/(J+xA)$ containing $I$, and so $\height(I/(J+xA)) = 0$. This completes the proof using the ideal $J+xA$.
\end{proof}
