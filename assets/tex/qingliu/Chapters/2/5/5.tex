\mtexe{2.5.5} 
\begin{proof}
	The argument is nearly identical to the previous; we only need to keep track of degrees. We can again assume $\frp_r$ contains no $\frp_i$ for $i < r$. So we can choose elements $y_i \in \frp_i \setminus \frp_r$. For each $i$, if $\frp_r$ contained all homogeneous components of $y_i$, then it would contain their sum $y_i$, which it doesn't. So, some homogeneous component of $y_i$ is not contained in $\frp_r$, but $\frp_i$ is homogeneous, so it does contain this component. Thus, without loss of generality, we can assume each $y_i$ is homogeneous by replacing it with this homogeneous component if necessary. Similarly, we can choose $z \in I \setminus \frp_r$ homogeneous. But then the product $a = z\cdot y_1 \cdots y_{r-1}$ is homogeneous and not in $\frp_r$ by primality. \\
	
	By the induction hypothesis, we can choose such a $b$. If $b \notin \bigcup_i \frp_i$ then we're done, so assume otherwise. Since $b \notin \frp_i$ for $i < r$, we must have $b \in \frp_r$. Then $a \notin \frp_r$, so $a^{\deg b} \notin \frp_r$ by primality, and so $a^{\deg b} + b^{\deg a} \notin \frp_r$. For $i < r$, we have $a \in \frp_i$ so $a^{\deg b} \in \frp_i$, but $b \notin \frp_i$, so $b^{\deg a} \notin \frp_i$ by primality. Then we again get $a^{\deg b}+b^{\deg a} \notin \frp_i$. So we have found a homogeneous element (of degree $(\deg a)(\deg b)$) of $I$ not contained in any $\frp_i$ as desired.
\end{proof}
