\mtexe{2.4.4} 
\begin{proof}
	We associate to $X$ a graph $G = (V,E)$ defined as follows: for each irreducible component $X_i$, we define a vertex $v_i \in V$, and if $X_i \cap X_j \neq \emptyset$, then $(v_i,v_j) \in E$. Now, the claim we are trying to show is equivalent to saying that $X$ is connected if and only if $G$ is a connected graph.
	
	First, suppose that $X$ is connected. To show that $G$ is connected, it suffices to show the following property: if $V = S \sqcup T$ is partitioned into any two nonempty subsets, then there is an edge $(s,t) \in E$ with $s \in S$ and $t \in T$. To see this, let $S,T$ be as stated, and let
	\[ W = \bigcup_{i \in S} X_i \text{ and } Z = \bigcup_{i \in T} X_i \]
	Since these are both finite nonempty unions and each $X_i$ is closed, we have that $W$ and $Z$ are nonempty and closed. But also we have that $X = W \cup Z$, and so by connectedness, these cannot be disjoint, i.e. $W \cap Z \neq \emptyset$. Any point in this intersection lies in some $X_i$ for $i \in S$ and some $X_j$ for $j \in T$. But then $X_i \cap X_j \neq \emptyset$ and $(i,j) \in E$ as desired.
	
	Conversely, suppose that $G$ is connected. Write $X = W \sqcup Z$ for two disjoint closed subsets of $X$; we wish to show that one of them is empty. For each $i$, note that $X_i = (X_i \cap W) \cup (X_i \cap Z)$, and so by irreducibility, either $X_i = X_i \cap W$ or $X_i = X_i \cap Z$. In other words, each irreducible component is contained in at least one of $W$ or $Z$. Let $S = \{ i : X_i \subseteq W \}$ and $T = \{ i : X_i \subseteq Z \}$. We've shown that $V = S \cup T$. In fact, $S$ and $T$ are disjoint, for if $i \in S \cap T$, then $X_i \subseteq W \cap Z = \emptyset$. Suppose both $S$ and $T$ are nonempty. From the above equivalent condition for connectedness of a graph, we have that there is some edge $(s,t)$ with $s \in S$ and $t \in T$. In other words, we have $X_s \cap X_t \neq \emptyset$. But we have $X_s \cap X_t \subseteq W \cap Z = \emptyset$. This is a contradiction, and so we cannot have both $S$ and $T$ nonempty; WLOG assume $S = \emptyset$. Then $T = V$, so $X_i \subseteq Z$ for all $i$, and so $Z = X$ and $W = \emptyset$, which is what we wished to show. \\
	
	Now, suppose $X$ is integral. We've already noted that this implies that $X$ is integral at $x$ for each $x \in X$. Further, we have that $X$ is irreducible, so in the language above, $V$ is a singleton. The graph on 1 vertex is certainly connected, and so $X$ is connected.
	
	Conversely, suppose $X$ is integral at each of its points and connected. Then $X$ is clearly reduced at each of its points since domains are reduced, and so it suffices to show that $X$ is irreducible. But note that being integral at each of its points implies that each $x \in X$ is contained in a unique irreducible component, and so in the language of graphs above, $E = \emptyset$. That is, $X_i \cap X_j = \emptyset$ for all $i,j$. But now, we have a connected graph with no edges, and so we must have $|V| \leq 1$, i.e. there is at most one irreducible component. This component must be all of $X$, and so $X$ is irreducible, and hence integral as desired.
\end{proof}
