\mtexe{2.4.6} 
\begin{proof}
	Note that it suffices to prove the equivalence of (i) and (ii). Indeed, suppose we have done so, let $X$ be a scheme, let $A = \Spec \scO_X(X)$, and apply the proven equivalence to the scheme $\Spec A$. Then we conclude $\Spec \scO_X(X) = \Spec A$ is connected iff $\scO_{\Spec(A)}(\Spec A) = A = \scO_X(X)$ has no nontrivial idempotents, which is precisely (ii) iff (iii).
	
	So now, suppose first that $X$ is disconnected. Then we can write $X = U \cup V$ for disjoint open sets $U$ and $V$. The sections $1 \in \scO_X(U)$ and $0 \in \scO_X(V)$ trivially agree on the overlap $U \cap V = \emptyset$. So we can glue them to a section $f \in \scO_X(X)$. Then $f \neq 0$ since the restriction to $U$ is nonzero, and similarly $f \neq 1$. But $f^2|_U = (f|_U)^2 = 1^2 = 1$ and $f^2|_V = (f|_V)^2 = 0^2 = 0$, so $f^2 = f$ shows that $\scO_X(X)$ has a nontrivial idempotent.
	
	Conversely, suppose that $e \in \scO_X(X)$ is a nontrivial idempotent. Then consider the two open subsets $X_e$ and $X_{1-e}$. Since $e(1-e) = e-e^2 = e-e = 0$, we cannot have $e_x$ and $(1-e)_x$ both be units in any stalk $\scO_{X,x}$. Hence $X_e$ and $X_{1-e}$ are disjoint. On the other hand, this same equation shows that one of $e,1-e$ must be contained in the unique maximal ideal of $\scO_{X,x}$. Finally, $e_x+(1-e)_x = 1$ is not contained in this maximal ideal, and so exactly one of them is contained in the maximal ideal. Thus one of them is a unit, whence $x$ is contained in at least one of $X_e$ and $X_{1-e}$. Thus we've written $X$ as the disjoint union of open subsets, so $X$ is disconnected. \\
	
	If $(A,\frm)$ is local, then any closed subset of $\Spec A$ is of the form $V(I)$, and $I \subseteq \frm$, so $\frm \in V(I)$. Thus there are no disjoint closed subsets and so $\Spec A$ is connected. \\
	
	Notice that under these assumptions, $U$ is clearly open and $X \setminus U$ is the union of all other connected components and so is also open. Thus, the uniqueness of $e$ is obvious, since we've specified its restriction to an open cover, and existence follows by gluing.
	
	The next claim is false in characteristic 2. For example, in $\Spec(\FF_2 \times \FF_2)$, $e = (1,0)$ is the claimed section, but it is not indecomposable, as it is the sum of the idempotents $(1,1)$ and $(0,1)$. Outside of characteristic 2, however, we are okay. Indeed, write $e$ as above as the sum of two idempotents $e = f+g$. Then $1 = e|_U = f|_U + g|_U$, but $U$ is connected, so the only idempotents are $0,1$. Thus, WLOG, $f|_U = 0$ and $g|_U = 1$. For each other connected component $V$, we get $0 = e|_V = f|_V + g|_V$, and so $f|_V = g|_V = 0$ since again $V$ is connected and $\scO_X(V)$ has characteristic $\neq 2$. So, gluing back gives $f = 0$ and $g = e$, and so $e$ is indecomposable.
	
	So, finally, we show that this association is bijective. It's clearly bijective, for if $e$ corresponds to $U$ and $f$ corresponds to $V$ for components $U \neq V$, then $e|_V = 0$ while $f|_V = 1$, so $e \neq f$. So we wish to show it is surjective; let $e \in \scO_X(X)$ be any nonzero indecomposable idempotent. Since it is nonzero, it cannot be zero when restricted to each connected component. So, there is some component $U$ with $e|_U \neq 0$. Since $U$ is connected and $e|_U$ is idempotent, we must have $e|_U = 1$. Now, suppose $e|_V \neq 0$ for some other component $V$. Then $e = (e-f) + f$, where $f$ is the idempotent corresponding to $V$ and $e-f,f$ are nontrivial. This contradicts indecomposability, and so we must have $e|_V = 0$ for each other component, whence $e$ is precisely the idempotent corresponding to $U$.
\end{proof}
