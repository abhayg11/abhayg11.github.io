\mtexe{2.3.10} 
\begin{proof}
	Let $f \in \scO_X(X)$ be a global section, and let $U_i = D_{+}(T_i)$ so that $U_i \cong \Spec A[T_0/T_i,\ldots,T_n/T_i]$. Fix some indices $i \neq j$. Then we have that the restriction of $f$ to $U_i$ can be expressed as a polynomial in the variables $T_0/T_i,\ldots,T_n/T_j$, so after clearing denominators, we have:
	\[ g = T_i^a(f|_{U_i}) \]
	for some $g \in A[T_0,\ldots,T_n]$ that is not a multiple of $T_i$. But similarly, we can clear denominators in $j$, i.e.
	\[ T_j^b(f|_{U_j}) \in A[T_0,\ldots,T_n] \]
	for some $b$. In the intersection, this implies:
	\[ T_j^bg = T_i^a\left(T_j^b(f|_{U_i \cap U_j})\right) \]
	which shows that $T_j^bg$ is a multiple of $T_i^a$ in $A[T_0,\ldots,T_n]$. By uniqueness of polynomial representation, we must have $a=0$ since $T_i$ divides neither $T_j$ nor $g$. But this means that $f|_{U_i} \in A$. Since $i$ is arbitrary, this shows that $f \in A$, and since $f$ was arbitrary, this gives that $\scO_X(X) = A$. \\
	
	Now, note that if $n = 0$, then $\PP_A^n = \Spec A$ is affine. Conversely, suppose that $\PP_A^n$ is affine, so it is isomorphic to $\Spec B$ for some ring $B$. But then $A = \scO_X(X) \cong \scO_{\Spec B}(\Spec B) = B$, so we have $\PP_A^n \cong \Spec A$. We want to show that this implies $n=0$. One approach (using material from later sections) is to reduce to the case for fields. Namely, choosing an arbitrary maximal ideal $\frm$ of $A$ (i.e. a closed point) with residue field $k = A/\frm$ gives the following fibre product:
	\[ \begin{tikzcd} \PP_k^n \arrow{r}{} \arrow{d}{} & \PP_A^n \arrow{d}{} \\ \Spec k \arrow{r}{} & \Spec A \end{tikzcd} \]
	where the right edge is the claimed isomorphism. But then the left arrow would also be an isomorphism, which is easily disproved by counting. That is, an isomorphism of schemes induces in particular a homeomorphism of topological spaces, but $\Spec k$ consists of a single point, while $\PP_k^n$ has at least two points for $n > 0$: both the zero ideal and the ideal generated by $T_0$ are homogeneous, prime, and do not contain the irrelevant ideal. So, we cannot have $n>0$ and we conclude $n=0$ as desired.
\end{proof}
