\mtexe{2.3.22} 
\begin{proof}
	It is clear that $n$ is an automorphism of $k[T]$ for each $n$, with inverse $-n$, and thus of $\AA^1_k$. Now, suppose that $U$ is an open subset of $\AA^1_k$. Then the complement of $U$ is closed, so it is of the form $V(I)$ for some ideal $I$, but since $k[T]$ is a PID, we can write $I = (f)$ for some polynomial $f$. Then $V(I)$ is precisely the set of prime factors of $f$, with $U$ being its complement. If $f \in k$ is a unit, then it has no prime divisors and $U = \AA^1_k$. If $f = 0$, then every prime divides it, and so $U = \emptyset$.
	
	Otherwise, $V(f)$ is nonempty and finite; in particular, it contains some prime ideal $(g(T))$. But if $(g(T+n)) \notin V(f)$, then $(g(T+n)) \in U$, and so $(g(T+n-n)) = (g(T)) \in U$, contrary to assumption. So, $V(f)$ contains $(g(T+n))$ for every $n$. Since we are in characteristic zero, $(g(T+n)) = (g(T+m))$ iff $n = m$, and so we have our contradiction, since $V(f)$ is supposed to be finite, but we have an injection $\ZZ \hookrightarrow V(f)$ given by $n \mapsto (g(T+n))$. \\
	
	Now, note that the ring homomorphism $\varphi : k \hookrightarrow k[T]$ induces a map $p : \AA^1_k \to \Spec k$. Further, the ring map is $\ZZ$-invariant, since $n \cdot \varphi(a) = \varphi(a)$, since $\varphi(a)$ is a constant and has no $T$-terms. So, $p \circ n = p$ for any $n \in \ZZ$. Further, $p$ has the universal property of the quotient map of schemes. Indeed, let $Z$ be any scheme and $f : \AA^1_k \to Z$ a morphism with $f \circ n = f$ for all $n \in \ZZ$. Then, pick any $z$ in the image of $f$, and pick an affine neighborhood $U$ of $z$ in $Z$. Then $f^{-1}(U)$ is a nonempty open subset of $\AA^1_k$, and I claim it is $\ZZ$-stable. For if $x \in f^{-1}(U)$, then $f(n(x)) = f(x) \in U$, and so $n(x) \in f^{-1}(U)$ as well. But by the previous claim, this forces $f^{-1}(U) = \AA^1_k$ to be the whole space.
	
	In other words, $f$ factors through a map $f : \AA^1_k \to U = \Spec A$ (which I'm also denoting by $f$). But this is precisely determined by a map $f^\# : A \to k[T]$, and the $\ZZ$-invariance implies that $f^\#(a)(T+n) = f^\#(a)(T)$ for all $n \in \ZZ$ and $a \in A$. In other words, each image is a constant polynomial, and so $f^\#$ has image in $k$, whence $f$ factors through $p$. But this is precisely what we wished to show. \\
	
	On the other hand, from exercise 2.2.14, we know that the quotient $\AA^1_k/\ZZ$ of ringed topological spaces exists and has underlying topological space homeomorphic to the quotient topological space $\AA^1_k/\ZZ$. In other words, choosing two elements $a,b \in k$ such that $a-b \notin \ZZ$, we have that the ideals $(T-a)$ and $(T-b)$ map to different points in the ringed topological space quotient; in particular the ringed topological space has more than one point. But the topological space of $\Spec k$ is a single point, and so the two are not the same as ringed topological spaces (or even as topological spaces).
\end{proof}
