\mtexe{3.2.13}
\begin{proof}
	We will show something slightly more general than the claim in this problem. Fix an algebraic extension $L$ of $k$. Then I claim that $X_L$ is reduced, irreducible, or connected if and only if $X_{k'}$ is reduced, irreducible, or connected, respectively, for each finite subextension $L/k'/k$.
	
	Let's begin with the more straightforward directions. Let $K$ be a finite subextension of $L/k$, and the corresponding base-change morphisms $X_L \to X_K \to X$. First, suppose that $X_L$ is reduced. To show that $X_K$ is reduced, it suffices to show this on an open cover. Let $\Spec A$ be an open affine of $X_K$, and note that then $\Spec(A \otimes_K L)$ is an open affine in $X_L$ containing the preimage of $\Spec A$. So, the above morphism corresponds to the ring homomorphism $A \to A \otimes_K L$. But this map is injective since $A$ is a (free, hence) flat $K$-algebra, and the codomain is reduced as it comes from an open subscheme of a reduced scheme, so $A$ must be reduced as desired.
	
	Now, suppose that $X_L$ is irreducible. But $\Spec L \to \Spec K$ is surjective, which is preserved by base change, so $X_L \to X_K$ is surjective, and the image of an irreducible topological space is again irreducible. So $X_K$ is irreducible as claimed. Similarly, if $X_L$ is connected, then so is $X_K$ since the image of a connected space is connected.
	
	We'll now address each converse. Suppose first that $X_L$ is not reduced. Again, this can be checked locally, so it must be the case that some stalk $\scO_{X_L,p}$ is not a reduced ring. Choose an open affine $\Spec A \subseteq X$ containing the image of $p$, so that $\Spec(A \otimes_k L)$ is an open affine neighborhood of $X_L$ containing $p$. Now $A \otimes_k L$ is not reduced (else each of its localizations would be reduced, including at $p$), so we can choose a nonzero nilpotent
	\[ f = \sum_{i=1}^n a_i \otimes b_i \]
	for some $n$, some $a_1,\ldots,a_n \in A$, and some $b_1,\ldots,b_n \in L$. But now each $b_i$ is algebraic over $k$ by design, and there are finitely many of them, so $K = k[b_1,\ldots,b_n]$ is a finite extension of $k$. I claim that $X_K$ is also not reduced. One of its affine neighborhoods is $\Spec(A \otimes_k K)$, but then this ring contains the element $f$. The map $A \otimes_k K \to A \otimes_k L$ is injective, so the fact that $f^N = 0$ in the latter ring means $f^N=0$ in the former one as well, so that $X_K$ is not reduced as claimed.
	
	Second, suppose that $X_L$ is not irreducible. If $X$ itself is not irreducible, we are done, so assume WLOG that $X$ is irreducible. Let $W_1,W_2$ be two distinct irreducible components of $X_L$, and endow them with the induced reduced subscheme structure. By Lemma 3.2.6, there exist finite subextensions $K_1,K_2$ of $L/k$ and reduced closed subschemes $Z_1 \subseteq X_{K_1}$ and $Z_2 \subseteq X_{K_2}$ such that $W_1 = (Z_1)_L$ and $W_2 = (Z_2)_L$. As noted in the proof of that lemma, we can replace $K_1,K_2$ with any of their finite extensions contained in $L$, and so choosing, for example, the compositum allows us to assume $K_1 = K_2 = K$. But now again using that $\Spec L \to \Spec K$ is surjective, we get that $W_i \to Z_i$ is surjective, and so each $Z_i$ is irreducible. Furthermore, I claim that each $Z_i$ is actually an irreducible component of $X_K$, which would complete the proof of this converse, since they are distinct (having different base changes in $X_L$). Indeed, by similar reasoning to previous problems in this section, the generic points of $X_L,X_K$ all make up the generic fiber. But the generic points of $W_1,W_2$ are generic points of $X_L$, so they map to the unique generic point in $X$, and they also map to the generic points of $Z_1,Z_2$. So, the generic points of $Z_1,Z_2$ lie in the generic fiber and so must be generic points of $X_K$, i.e. their closures in $X_K$ are full irreducible components.
	
	Finally, suppose that $X_L$ is not connected. Then $\scO_{X_L}(X_L)$ has a nontrivial idempotent $f$. But since $k \to L$ is flat and $X$ is a Noetherian $k$-scheme, proposition 3.1.24 tells us that
	\[ \scO_{X_L}(X_L) \cong \scO_X(X) \otimes_k L \]
	Under this isomorphism, we may write $f = \sum_{i=1}^n f_i \otimes b_i$ for some $f_1,\ldots,f_n \in \scO_X(X)$ and $b_1,\ldots,b_n \in L$. As in the reduced case, we can then consider $K = k[b_1,\ldots,b_n]$ and conclude that
	\[ f \in \scO_X(X) \otimes K \cong \scO_{X_K}(X_K) \]
	so that $X_K$ is also disconnected. This completes the proof, and of course, taking $L = \bar{k}$ handles the original problem.
\end{proof}
