\mtexe{3.2.10}
\begin{proof}
	Recall that we previously showed (exercise 2.3.20) that if $A$ is a ring, $G$ is a finite group of automorphisms of $A$, $A^G$ is the invariant subring, and $p : \Spec A \to \Spec A^G$ the morphism induced by $A^G \hookrightarrow A$, then $p(x_1) = p(x_2)$ if and only if there is a $\sigma \in G$ such that $\sigma(x_1) = x_2$.
	In our case, $G$ is the Galois group of $K/k$ and $A = L \otimes_k K$; in order to show that $G$ acts transitively on $\Spec A$, it thus suffices to show that $p(x_1) = p(x_2)$ for any $x_1,x_2 \in A$. So, we should compute $A^G = (L \otimes_k K)^G$. For this, consider the exact sequence
	\[ 0 \to k \to K \xrightarrow{f} \prod_{\sigma \in G} K \]
	of $k$-modules, where $f(a) = (\sigma(a)-a)_{\sigma \in G}$. Indeed, this is exact since the kernel of the nontrivial map is precisely those $a \in K$ such that $\sigma(a) = a$ for all $\sigma$; i.e. it is the fixed field of $K$ under $G$, which is $k$ by definition. Then, $L$ is a free $k$-module, hence a flat $k$-module, so tensoring gives the exact sequence
	\[ 0 \to L \to K \otimes_k L \xrightarrow{f \otimes \id_L} \to \left(\prod_{\sigma \in G} K\right) \otimes_k L \]
	So, $L$ is precisely the kernel of $f \otimes \id_L$. But this map acts exactly as $G$ does: for a simple tensor $a \otimes b \in K \otimes_k L$, we have $\sigma(a \otimes b) = \sigma(a) \otimes b = (\sigma \otimes \id_L)(a \otimes b)$, and so the kernel of $f \otimes \id_L$ is precisely the fixed subring of $K \otimes_k L$. Note that this proof only requires $L$ to be a $k$-module; nowhere did we use that it is a field.
	
	But now, we're done, since $p(x_1) = p(x_2)$ is the unique point of $\Spec L$ for any $x_1,x_2$ as above. \\
	
	Note that the action of $G$ on $X_K$ is as follows: each element $\sigma \in G$ induces a map which we also denote $\sigma : \Spec K \to \Spec K$. Thus we get a double Cartesian diagram
	\[ \begin{tikzcd} X_K \arrow[r, "f_\sigma"] \arrow[d] & X_K \arrow[r] \arrow[d] & X \arrow[d] \\ \Spec K \arrow[r, "\sigma"] & \Spec K \arrow[r] & \Spec k \end{tikzcd} \]
	The map $f_\sigma$ is the action of $\sigma$ on $X_K$. This description makes it clear that, on points, each $f_\sigma$ is a homeomorphism with inverse $f_{\sigma^{-1}}$, and so carries irreducible components to irreducible components. So, it suffices to understand each $f_\sigma$ on generic points of $X_K$.
	
	First, let $\eta \in X$ denote its generic point. Then, I claim that each generic point of $X_K$ lies in the fiber over $\eta$. Indeed, let $\xi \in X_K$ be a generic point of some irreducible component of $X_K$, and let $\pi : X_K \to X$ denote the projection. Choose an affine neighborhood $\Spec A$ of $\pi(\xi)$ and an affine neighborhood $\Spec B$ of $\xi$ contained in $\pi^{-1}(\Spec A)$. Then $\eta \in \Spec A$ is the unique minimal prime, and $\xi \in \Spec B$ is one of its minimal primes. Note that $k \hookrightarrow K$ is free, hence flat, and that this is preserved by base change, so $A \to B$ is flat. So, this extension satisfies going down; in particular, $\pi(\xi)$ is a prime ideal in $\Spec A$, so it contains $\eta$, and if this containment were proper, we could find a prime ideal properly contained in $\xi$ that projects to $\eta$. But $\xi$ is minimal, so this cannot be. Hence, $\pi(\xi) = \eta$ as claimed.
	
	So now, we know that all generic points lie in the fiber $(X_K)_\eta$ over $\eta$, which can be constructed itself as a fiber product via the following diagram:
	\[ \begin{tikzcd} (X_K)_\eta \arrow[r] \arrow[d] & \Spec k(\eta) \arrow[d] \\ X_K \arrow[r,"\pi"] \arrow[d] & X \arrow[d] \\ \Spec K \arrow[r] & \Spec k \end{tikzcd} \]
	But this makes it clear that $(X_K)_\eta \cong \Spec(k(\eta)) \times_k \Spec K = \Spec(k(\eta) \otimes_k K)$, and the first part of this problem demonstrates that $G$ therefore acts transitively on it, as claimed. This holds for all Galois extensions $K/k$, and so it holds for $k^{sep}/k$, the separable closure of $k$ in a fixed algebraic closure $\bar{k}$. Finally, the base change to $\bar{k}$ induces a homeomorphism $X_{\bar{k}} \to X_{k^{sep}}$, so they have the same irreducible components, which must therefore have equal dimension. \\
	
	Note that there are only finitely many connected components of $X_K$ since it is of finite type over a field. So, connected components are both open and closed. Let $U \subseteq X_K$ be a connected component of $X_K$, and define
	\[ Y = \bigcup_{\sigma \in G} \sigma(U) \]
	Then, noting that $\sigma$ is a homeomorphism, $Y$ is open as the union of open sets and closed as the finite union of closed sets. So, writing $Z = X_K \setminus Y$ splits $X_K$ as a disjoint union of closed sets. But now $Y$ and $Z$ are closed $G$-invariant subsets of $X_K$. I claim that this means that they are of the form $\pi^{-1}(Y'),\pi^{-1}(Z')$, respectively, for closed subsets $Y',Z' \subseteq X$, where $\pi : X_K \to X$ is the canonical projection. First, note that this claim would complete the proof. Indeed, $\Spec K \to \Spec k$ is surjective, so $\pi$ is surjective. Hence, for $x \in X$, $x = \pi(t)$ for some $t \in X_K = Y \cup Z$. So, $x \in Y' \cup Z'$. Further, $Y'$ and $Z'$ are disjoint. Indeed, if $x \in Y' \cap Z'$, then the fiber over $x$ intersects both $Y$ and $Z$ nontrivially. But $G$ acts transitively on this fiber (by the same argument as above), so if one point of the fiber is in $Y$, every point of the fiber must be in $Y$ since $Y$ is $G$-invariant. Thus, we have written $X = Y' \sqcup Z'$ for closed $Y'$ and $Z'$, and $Y'$ is nonempty since $Y$ is nonempty, so by connectedness of $X$, we must have $Z' = \emptyset$, whence $Z = \emptyset$. So $Y = X_K$ and we see that all connected components are in a single orbit, as desired.
	
	So, we are reduced to showing that if $Y \subseteq X_K$ is closed and $G$-invariant, then it is the preimage of some closed $Y' \subseteq X$. It suffices to show this when $X$ is affine, since we can cover $X$ by affine open neighborhoods, the $G$-orbit of a point of $X_K$ is contained in the preimage of such a neighborhood, and being a preimage and closed can be computed locally. So, suppose $X = \Spec A$, whence $X_K = \Spec(A \otimes_k K)$, so $Y = V(I)$ for some ideal $I \subseteq A \otimes_k K$. WLOG, we can assume $I = \sqrt{I}$. We seek an ideal $J \subseteq A$ such that $V(I) = V(J(A \otimes_k K))$, and since $I$ is radical, this is equivalent to $I = \sqrt{J(A \otimes_k K)}$. Choose $J = I \cap A$, so one containtment is clear. For the other, note that if $x \in I$, then
	\[ \prod_{\sigma \in G} (T - \sigma(x)) \in A[x] \]
	since each coefficient is in the fixed subring $(A \otimes_k K)^G = A$ (proven above). But since $x \in I$ and $I$ is $G$-invariant, $\sigma(x) \in I$ for all $\sigma$, so that each coefficient is in $I$ as well. So, each coefficient is in $I \cap A = J$, and since $x$ is a root, we can solve for the leading term to get $x^n$ as a sum of terms of the form $-a_ix^i$, which is in $J(A \otimes_k K)$. So $x \in \sqrt{J(A \otimes_k K)}$ as claimed, completing the argument.
\end{proof}
