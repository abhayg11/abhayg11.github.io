\mtexe{3.2.9}
\begin{proof}
	Let $x \in X$ be a closed point. Then if $\Spec A$ is an open affine neighborhood of $x$, then $A$ must be a finitely generated $k$-algebra, and we can compute the residue field $k(x)$ by localizing $A$ at $x$ and quotienting by the image of the maximal ideal. But localization commutes with quotients, so equivalently we can quotient and then localize. But since $x$ is closed, it corresponds to a maximal ideal, so the quotient is a field, and the localization does nothing. That is, $k(x) \cong A/x$, where $x$ on the right denotes the maximal ideal. But then $k(x)$ is a finitely generated $k$-algebra and a field, so by Zariski it is a finite field extension of $k$, and so embeds (in some way) into $\bar{k}$. This embedding gives a map $\Spec\bar{k} \to \Spec k(x)$, which we can compose with the canonical map $\Spec k(x) \to X$ to get an element of $h \in X(\bar{k})$.
	
	Now, by assumption, $f$ and $g$ induce the same map $X(\bar{k}) \to Y(\bar{k})$, and so $f \circ h = g \circ h$ (as morphisms of schemes). But $h(\ast) = x$, where $\ast$ is the unique point of $\Spec\bar{k}$, and so this gives $f(x) = f(h(\ast)) = g(h(\ast)) = g(x)$ as desired. I think there's an error in the next claim (that they agree on all points), unless we assume $Y$ is separated over $k$. Roughly, the argument should be the analog of ``$f$ and $g$ agree on a dense set and $Y$ is Hausdorff, so $f=g$.'' In this case, the proper substitutions are that the set of closed points is indeed dense in $X$, and separatedness guarantees that therefore $f=g$. Indeed, then the argument is as follows:
	
	Since $Y/k$ is separated, $\Delta : Y \to Y \times_k Y$ is a closed immersion. Consider the base change along $X \to Y \times_k Y$ given by $x \mapsto (f(x),g(x))$ -- that is, the morphism induced by the maps $f$ and $g$ -- and denote it $u : K \to X$. Closed immersions are stable under base change, so $u$ is a closed immersion. Hence, the image of $u$ is a closed subset of $X$. On the other hand, I claim the image of $u$ contains all points $x \in X$ such that $f(x) = g(x)$. Indeed, if $f(x) = g(x)$, then $x$ lies over the pair $(f(x),g(x))$ in $Y \times_k Y$, and $\Delta(f(x)) = (f(x),f(x)) = (f(x),g(x))$ as well. But then there is a point of $K$ that maps to $f(x) \in Y$ and $x \in X$, as desired.
	
	Now, the image of $u$ is a closed subset of $X$ containing all closed points of $X$. So, we are done if the set of closed points is dense. It suffices to show this locally, so by passing to an open affine, we can assume $X = \Spec A$. Then, we can equivalently show that each set in a base for the topology contains a closed point, so we would like to show that if $D(f)$ is nonempty, it contains a maximal ideal. Since $D(f)$ is nonempty, it is not nilpotent, so $A_f$ is nontrivial and has a maximal ideal. Its preimage $\frp$ is a prime in $D(f)$, and so it suffices to show that it is maximal. But $A/\frp$ is a domain, and localizing at $f$ gives $(A/\frp)_f \cong A_f/\frp_f$, which is a field by assumption, as $\frp_f$ was chosen maximal. But then $A_f$ is also a finitely generated $k$-algebra, as the image of $A[t]$ under $t \mapsto 1/f$, and so by Zariski $A_f/\frp_f$ is a finite $k$-vector space. But then $A/\frp$ is also a finite $k$-vector space, as a subspace of $A_f/\frp_f$, and so must actually be a field.
\end{proof}
