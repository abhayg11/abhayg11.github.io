\mtexe{3.2.9}
\begin{proof}
	Let $x \in X$ be a closed point. Then if $\Spec A$ is an open affine neighborhood of $x$, then $A$ must be a finitely generated $k$-algebra, and we can compute the residue field $k(x)$ by localizing $A$ at $x$ and quotienting by the image of the maximal ideal. But localization commutes with quotients, so equivalently we can quotient and then localize. But since $x$ is closed, it corresponds to a maximal ideal, so the quotient is a field, and the localization does nothing. That is, $k(x) \cong A/x$, where $x$ on the right denotes the maximal ideal. But then $k(x)$ is a finitely generated $k$-algebra and a field, so by Zariski it is a finite field extension of $k$, and so embeds (in some way) into $\bar{k}$. This embedding gives a map $\Spec\bar{k} \to \Spec k(x)$, which we can compose with the canonical map $\Spec k(x) \to X$ to get an element of $h \in X(\bar{k})$.
	
	Now, by assumption, $f$ and $g$ induce the same map $X(\bar{k}) \to Y(\bar{k})$, and so $f \circ h = g \circ h$ (as morphisms of schemes). But $h(\ast) = x$, where $\ast$ is the unique point of $\Spec\bar{k}$, and so this gives $f(x) = f(h(\ast)) = g(h(\ast)) = g(x)$ as desired. I think there's an error in the next claim (that they agree on all points), unless we assume $Y$ is separated over $k$. Roughly, the argument should be the analog of ``$f$ and $g$ agree on a dense set and $Y$ is Hausdorff, so $f=g$.'' In this case, the proper substitutions are that the set of closed points is indeed dense in $X$, and separatedness guarantees that therefore $f=g$. Indeed, then the argument is as follows:
	
	Since $Y/k$ is separated, $\Delta : Y \to Y \times_k Y$ is a closed immersion. Consider the base change along $X \to Y \times_k Y$ given by $x \mapsto (f(x),g(x))$ -- that is, the morphism induced by the maps $f$ and $g$ -- and denote it $u : K \to X$. Closed immersions are stable under base change, so $u$ is a closed immersion. Hence, the image of $u$ is a closed subset of $X$. On the other hand, I claim the image of $u$ contains all points $x \in X$ such that $f(x) = g(x)$. Indeed, if $f(x) = g(x)$, then $x$ lies over the pair $(f(x),g(x))$ in $Y \times_k Y$, and $\Delta(f(x)) = (f(x),f(x)) = (f(x),g(x))$ as well. But then there is a point of $K$ that maps to $f(x) \in Y$ and $x \in X$, as desired.
	
	Now, the image of $u$ is a closed subset of $X$ containing all closed points of $X$. So, we are done if the set of closed points is dense. It suffices to show this locally, so by passing to an open affine, we can assume $X = \Spec A$. The closure of the set of maximal ideals is $V(I)$ for some ideal $I$. This $I$ must be contained in each maximal ideal, so
	\[ I \subseteq \bigcap_{\frm \in X \atop \frm \text{ is maximal}} \frm = \sqrt{(0)} \]
	since $A$ is a finitely generated $k$-algebra. But then $V(I) \supseteq V(\sqrt{(0)}) = X$, so we're done. \\
	
	Now, we would like to show that $f = g$ as morphisms of schemes. Again, it suffices to show this locally since we can then glue the morphisms back uniquely. So, for $x \in X$, choose an affine neighborhood $\Spec B$ of $f(x) = g(x)$ in $Y$, take the preimage to get an open subset of $X$, and reduce to an affine open neighborhood $\Spec A$ contained in this open set containing $x$. It suffices to show that $f = g$ when restricted to $\Spec A \to \Spec B$, and so we may assume $X = \Spec A$ and $Y = \Spec B$. Further, $B$ is a finitely generated $k$-algebra, say $B = k[y_1,\ldots,y_n]/I$ for some ideal $I$, which gives a closed immersion $\iota : \Spec B \hookrightarrow \AA_k^n$. But closed immersions are monomorphisms, so showing that the compositions $\iota \circ f = \iota \circ g$ would imply that $f = g$. Hence, we may assume that $Y = \AA_k^n$.
	
	Finally, consider the base change to $\Spec(\bar{k})$. The maps $f_{\bar{k}}$ and $g_{\bar{k}}$ are equal by assumption, and so we have the following commutative diagram:
	\[ \begin{tikzcd} X_{\bar{k}} \arrow[r] \arrow[d] & X \arrow[d,shift left,"f"] \arrow[d, shift right,"g"] \\ \AA_{\bar{k}}^n \arrow[r] & \AA_k^n \end{tikzcd} \]
	So, we would be done if the top arrow is an epimorphism. This is a map of affine schemes, so it also suffices to show that the corresponding map of rings is injective. But this is the map $A \to A \otimes_k \bar{k}$, which is injective since $A$ is a free, and so flat, $k$-module. This completes the suggested reductions. \\
	
	Now, we have two maps between affine schemes, which thus correspond to maps $\varphi,\psi : k[y_1,\ldots,y_n] \to A$ for some finitely generated $k$-algebra $A$. We would like to show that the maps on $\Spec$ are equal, which will follow if $\varphi = \psi$ itself. Exercise 2.3.7 also tells us how to understand the maps on $\Spec$ for $k$-points, but since $k$ is algebraically closed, every closed point is rational. So, we can conclude that for each maximal ideal $\frm$ of $A$, that $\varphi(y_i) + \frm = \psi(y_i) + \frm$ in the quotient $A/\frm \cong k$. I.e. $\varphi(y_i) - \psi(y_i) \in \frm$ for all $i$ and all $\frm$. But again, $A$ is a finitely generated $k$-algebra, so this shows that $\varphi(y_i) - \psi(y_i) \in \sqrt{(0)}$ for all $i$. But $X = \Spec A$ is geometrically reduced and $k = \bar{k}$, so $A$ is reduced, whence $\sqrt{(0)} = (0)$. So $\varphi(y_i) = \psi(y_i)$ for all $i$, and so $\varphi = \psi$.
\end{proof}
