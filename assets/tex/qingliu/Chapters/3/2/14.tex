\mtexe{3.2.14}
\begin{proof}
	Let $X$ be a geometrically reduced $k$-variety and $K/k$ an arbitrary field extension. To show that $X_K$ is reduced, it suffices to show that $X_{\bar{K}}$ is reduced for a fixed algebraic closure of $K$. So, we may assume $K$ is algebraically closed. Then $K$ contains an algebraic closure $\bar{k}$ of $k$ and $X_{\bar{k}}$ is also geometrically reduced, so we may also assume $k$ is algebraically closed. Finally, the argument of the previous problem shows that we may also assume $K$ is finitely generated over $k$, since a nilpotent arises from a finite sum (of course, we lose the assumption that $K$ is algebraically closed here).
	
	Note now that we can write $K = k(a_1,\ldots,a_n)$ for some elements $a_i \in K$. Then $K$ contains the subring $R = k[a_1,\ldots,a_n]$ and $Y = \Spec(R)$ is then an integral $k$-variety with $K(Y) = K$. Since $Y$ is integral, it is reduced, and since $k$ is algebraically closed, $Y$ is geometrically reduced. So, by proposition 3.2.15, $K(Y) = K$ is a finite separable extension of a purely transcendental extension $k(T_1,\ldots,T_m)$. Now, to show that $X_K$ is reduced, it suffices to show that base-changing to a simple transcendental extension preserves reducedness, since we already know that base-changing to an algebraic separable extension preserves reducedness. In other words, we want to show that $X_{k(t)}$ is reduced whenever $X$ is reduced over an arbitrary field $k$.
	
	For this, if $X_{k(t)}$ is not reduced, it fails to be reduced at some point, and projecting back to $X$, choosing a neighborhood, and pulling back converts the problem to showing that if $A$ is a reduced finitely generated $k$-algebra, then $A \otimes_k k(t)$ is reduced. We can mimic the proof of Proposition 3.2.7. Namely, note that $A$ is Noetherian, so it has finitely many minimal primes $\frp_1,\ldots,\frp_n$, and so the injection $A \hookrightarrow \bigoplus_i (A/\frp_i)$ gives an injection $A \otimes_k k(t) \hookrightarrow \bigoplus_i (A/\frp_i) \otimes_k k(t)$. To show that the former is reduced, it suffices to show that the latter is reduced, for which is suffices to show each summand is reduced. Hence, we may assume that $A$ is a domain. But then $A \otimes_k k(t)$ injects into $\Frac(A) \otimes_k k(t)$....
\end{proof}
