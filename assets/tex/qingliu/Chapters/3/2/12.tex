\mtexe{3.2.12}
\begin{proof}
	Throughout, fix an algebraic closure $\overline{K(X)}$ of $K(X)$, let $\bar{k}$ be the algebraic closure of $k$ in $\overline{K(X)}$, and let $k^{sep}$ be the separable closure of $k$ in $\bar{k}$.
	
	We are trying to compute the number of irreducible components of $X_{\bar{k}}$. First, we note that as usual, each generic point of $X_{\bar{k}}$ lies in the generic fiber of $\pi : X_{\bar{k}} \to X$. Indeed, if $\eta \in X$ is the generic point, and $\xi \in X_{\bar{k}}$ is some generic point, then choosing an affine neighborhood $\Spec A$ of $\pi(\xi)$, we get an affine neighborhood $\Spec(A \otimes_k \bar{k})$ of $\xi$. Since $k \hookrightarrow \bar{k}$ is free, it is flat, and so $A \to A \otimes_k \bar{k}$ is flat, whence it satisfies going-down. So, if $\pi(\xi)$ isn't minimal, then we can find a prime contained in $\xi$ that does contract to a minimal prime of $A$, but $\xi$ is minimal, so this is impossible. So $\pi(\xi)$ is a minimal prime, and $\eta$ is the generic point, so it is contained in $A$ and is the unique minimal prime there. So $\pi(\xi) = \eta$ as claimed.
	
	Further, we have that the generic fiber is precisely composed of generic points. That is, if $\pi(\xi) = \eta$, then $\xi$ is a generic point of $X_{\bar{k}}$. Indeed, choose affine neighborhoods as above, and note that $k \hookrightarrow \bar{k}$ is also integral, which means that $A \to A \otimes_k \bar{k}$ is also integral. Further, $A$ is also a free $k$-module, so $A$ is flat, which means this map is also an injection. So, $A \otimes_k \bar{k}$ also satisfies incomparability over $A$. In particular, if $\xi$ were not a minimal prime of $A \otimes_k \bar{k}$, then it contains a minimal prime, but both of these must then contract to $\eta$, contradicting incomparability. So we must have that $\xi$ is minimal, i.e. it is a generic point of $X_{\bar{k}}$.
	
	So, we have reduced the problem to counting the size of the fiber over $\eta$. By considering the diagram
	\[ \begin{tikzcd} \Spec(\bar{k} \otimes_k K(X)) \arrow[r] \arrow[d] & X_{\bar{k}} \arrow[r] \arrow[d,"\pi"] & \Spec \bar{k} \\ \Spec K(X) \arrow[r] & X \arrow[r] & \Spec k \end{tikzcd} \]
	we see that each square is cartesian, and so the fiber is isomorphic (as a scheme) to $\Spec(\bar{k} \otimes_k k(X))$, so we want to count this set.
	
	To do this, let $L = k^{sep} \cap K(X)$. Then we have a tower of extensions $K(X)/L/k$, and we have that $\bar{k}$ is a $k$-scheme, so base-changing along both of these extensions gives a diagram
	\[ \begin{tikzcd} Y \arrow[r] \arrow[d] & \Spec K(p) \arrow[d] \\ \Spec(\bar{k} \otimes_k K(X)) \arrow[r,"f"] \arrow[d] & \Spec(\bar{k} \otimes_k L) \arrow[r] \arrow[d] & \Spec \bar{k} \\ \Spec K(X) \arrow[r] & \Spec L \arrow[r] & \Spec k \end{tikzcd} \]
	We've also included in the diagram the fiber $Y$ of $f$ at some point $p \in \Spec(\bar{k} \otimes_k L)$. The reason for this is that we can compute the size of $\Spec(\bar{k} \otimes_k K(X))$ by summing over these fibers, so it suffices to find $Y$ and sum over $p$. First, note that $K(p)$, the residue field at $p$, can be computed as the field of fractions of $(\bar{k} \otimes_k L)/p$, and so is the field of fractions of a finitely generated, algebraic, separable domain over $\bar{k}$. Hence it must be contained in $\bar{k}$ and of course, contains $\bar{k}$, i.e. it is $\bar{k}$. But then $Y = \Spec(K(X) \otimes_L \bar{k}) = \Spec(K(X) \otimes_L \bar{L})$. By Corollary 3.2.14(d), $K(X)$ is geometrically irreducible over $L$, so $Y$ is irreducible. But also, $L \hookrightarrow \bar{k}$ is integral, so $K(X) \to K(X) \otimes_L \bar{k}$ is integral, and so $\dim(Y) = \dim(K(X)) = 0$. But then $Y$ is irreducible and zero-dimensional, so it is a single point. So, finally, we've reduced to computing the size of $\Spec(L \otimes_k \bar{k})$.
	
	This we can, at last, do explicitly. By the primitive element theorem, $L = k[T]/(f(T))$ for some irreducible separable polynomial $f \in k[T]$ of degree $n = [L:k]$. Thus,
	\[ L \otimes_k \bar{k} = (k[T]/f(T)) \otimes_k \bar{k} = \bar{k}[T]/(f(T)) \cong \bar{k}^n \]
	where the last isomorphism comes from the chinese remainder theorem and the fact that $f$ splits completely into distinct factors in $\bar{k}[T]$ by definition. So, it has exactly $n$ prime ideals of the form
	\[ \bar{k}[T] \oplus \cdots \oplus \bar{k}[T] \oplus 0 \oplus \bar{k}[T] \oplus \cdots \oplus \bar{k}[T] \]
	So, at last, we're done and have exactly $n = [L:k] = [K(X) \cap k^{sep} : k]$ irreducible components as claimed.
\end{proof}
