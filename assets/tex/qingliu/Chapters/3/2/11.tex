\mtexe{3.2.11}
\begin{proof}
	We'd like to show that $X_{\bar{k}}$ is connected, where $\bar{k}$ is a fixed algebraic closure of $k$. It suffices to show that $X_{k^{sep}}$ is connected, since these spaces are homeomorphic. If it isn't connected, we'd be able to find a finite subextension $K$ such that $X_K$ is disconnected, and without loss of generality, we can choose $K/k$ to be a normal extension. So, it suffices to show that $X_K$ is connected for all Galois extensions $K/k$. Let $G$ be the galois group of $K/k$, and recall that we can canonically identify $X(K)^G$ with $X(k)$. So, since $X(k) \neq \emptyset$, we can find a point $x \in X(K)^G$. I claim that $x$ must be contained in every connected component. Indeed, $x$ is in some connected component $U$, and by the previous problem, $G$ acts transitively on the connected components. So, if $V$ is some other connected, component, then $\sigma(U) = V$ for some $\sigma \in G$ but $\sigma(x) = x$ since it is $G$-invariant, and so $x \in V$ as well. But we also know that distinct connected components are disjoint, so this is a contradiction unless $X_K$ only has a single connected component. I.e. $X_K$ must be connected, which is what we wished to show. \\
	
	Note that this proof works almost verbatim for irreducibility, except that distinct irreducible components need not be disjoint. So, our counterexample should be one where all irreducible components contain a $k$-point of $X$. Consider $X = \Spec A$ where $A = \RR[x,y]/(x^2+y^2)$ and $k = \RR$. Then $X$ is irreducible since $x^2+y^2$ is irreducible in $\RR[x,y]$ and $X(k) \neq \emptyset$ since $A$ has a maximal ideal with quotient $k$ -- namely, the ideal $(x,y)$. However, taking $\bar{k} = \CC$, we have $X_K = \CC[x,y]/(x^2+y^2) \cong \CC[x,y]/(x+iy) \oplus \CC[x,y]/(x-iy) \cong \CC[x] \oplus \CC[x]$, which is not irreducible.
\end{proof}
