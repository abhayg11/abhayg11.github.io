\mtexe{3.1.9}
\begin{proof}
	More generally, note that if $A,B$ are $R$-algebras with multiplicative subsets $S,T$, respectively, then $S^{-1}A \otimes_R T^{-1}B \cong (S \otimes T)^{-1}(A \otimes_R B)$, where $S \otimes T = \{ s \otimes t : s \in S,t \in T \}$. In our case, this gives that $k(u) \otimes_k k(v)$ is the localization of $k[u] \otimes_k k[v] \cong k[u,v]$ at $T' = T \otimes T$, where $T$ is the nonzero elements of $k[u],k[v]$, respectively. Under the above isomorphism, then, we have that a general element $P \otimes Q \in T'$ corresponds to the polynomial $P(u)Q(v) \in k[u,v]$, with both $P,Q$ nonzero, exactly as claimed. \\
	
	Let $\frm \in k[u,v]$ be maximal, and suppose for contradiction that there is no nonzero polynomial $P(u) \in \frm$. In particular, this shows that $\frm$ is disjoint from $T = \{ P(u) \in k[u,v] : P \neq 0 \}$, which is a multiplicative subset of $k[u,v]$. So, $\frm$ corresponds to a maximal ideal $\frm'$ of $T^{-1}k[u,v] = k(u)[v]$. So, $k(u)[v]/\frm'$ is a field extension of $k(u)$, and hence is transcendental over $k$. On the other hand, $k[u,v]/\frm$ is a field extension of $k$, and it is finitely generated as a $k$-algebra, so it is in fact a finite extension (Zariski Lemma). Localization commutes with taking quotients, so we would then get that $T^{-1}(k[u,v]/\frm)$ is the localization of a field at a multiplicative subset, which must therefore be just $k[u,v]/\frm$ itself, and so we have our contradiction as this is supposed to be both a finite extension of $k$ and transcendental over $k$. As noted, we thus conclude that $T' \cap \frm \neq \emptyset$. \\
	
	Now, suppose that $\frm$ is maximal in $A = T'^{-1}k[u,v]$. Then $\frm$ corresponds to some prime ideal $\frp$ of $k[u,v]$ that is disjoint from $T'$. Let's consider the different cases for $\height(\frp)$. Since $\dim(k[u,v]) = 2$, we must have $\height(\frp) \leq 2$. If $\height(\frp) = 2$, then $\frp$ is maximal, but then we've just shown that $\frp$ is not disjoint from $T'$. If $\height(\frp) = 0$, then $\frp = (0)$ itself, in which case it corresponds to $\frm = 0A$. This is a contradiction since $(0)$ is not maximal. So, we conclude that $\height(\frp) = 1$, in which case $\frp = gk[u,v]$ since $k[u,v]$ is a UFD. In order for this to be prime, we must have that $g$ is irreducible, and in order for it to be disjoint from $T'$, we must at least have that $g \notin k[u] \cup k[v]$. Then, it is clear that $\frm = gA$ is of the desired form.
	
	Conversely, suppose that $g \in k[u,v] \setminus (k[u] \cup k[v])$ is irreducible. Suppose, for contradiction, that $T' \cap gk[u,v] \neq \emptyset$. In other words, we may write $P(u)Q(v) = g(u,v)h(u,v)$ for some $P,Q,h$. But now, $g$ is irreducible and $k[u,v]$ is a UFD, so $g$ must be a factor of $PQ$, which means (WLOG) it is a factor of $P$. But $k[u]$ is also a UFD, so we can factorize $P$ in $k[u]$, showing that each of its irreducible factors is in $k[u]$ as well, giving $g \in k[u]$, contrary to assumption. So, the two sets are indeed disjoint, which gives that $gA$ is prime in $A$. But by again considering the height, it must actually be maximal: if $gA \subsetneq \frm$ in $A$, then this corresponds to $gk[u,v] \subsetneq \frm'$ in $k[u,v]$ disjoint from $T'$, but then $\frm'$ is a height 2 prime of $k[u,v]$, which is thus maximal and cannot be disjoint from $T'$. So we're done. \\
	
	We've already seen repeatedly that $\dim(A) = 1$. To see that $\Spec(A)$ is infinite, note that for each $n \geq 1$, $g_n(u,v) = u-v^n$ is irreducible and not in $k[u] \cup k[v]$. So, each $g_nA$ is maximal in $A$. It only remains to show they are distinct, but since this is true in $k[u,v]$, it is true in $A$ as well.
\end{proof}
