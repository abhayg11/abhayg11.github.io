\mtexe{3.1.11}
\begin{proof}
	Let $B = k[x_0,\ldots,x_n]$ and $C = k[y_1,\ldots,y_n]$ with the usual gradings. Consider the map $\varphi : C \to B$ given by $y_i \mapsto x_i$. Note that this is a homomorphism of graded rings. Let $C_{+}$ be as usual and let $M = \varphi(C_{+})B = (x_1,\ldots,x_n)$. Then by Lemma 2.3.40, $\varphi$ induces a morphism $p : \Proj(B) \setminus V_{+}(M) \to \Proj(C)$. But $V_{+}(M) = \{z\}$, since $z = (1,0,\ldots,0)$ in homogeneous coordinates means that $z$ corresponds to the prime ideal generated by $1x_i-0x_0 = x_i$ for $1 \leq i \leq n$, and $z$ is maximal among ideals not containing $B_{+}$.
	
	In homogeneous coordinates, if we have a point $(a_0,\ldots,a_n)$, it corresponds to the ideal generated by all $a_ix_j-a_jx_i$, which contracts under $\varphi$ to the ideal generated by all $a_iy_j-a_jy_i$ for $i,j \geq 1$, i.e. the point $(a_1,\ldots,a_n)$ in coordinates on $\PP_k^{n-1}$. These computations work in any field and in particular to $\overline{k}$ after base-changing. \\
	
	We will show that if $X$ is a closed subset of $\Proj B$ not containing $z$, then it cannot contain the fiber $p^{-1}(y)$ for any $y \in \Proj C$. In this case, $y$ is a prime ideal generated by homogeneous polynomials $f_i(y_1,\ldots,y_n)$ for some $1 \leq i \leq m$. Then, the ideal generated by $f_i(x_1,\ldots,x_n)$ in $\Proj B$ maps to $y$. If $X$ were to contain this ideal, it would contain any prime containing it, since $X$ is closed, but clearly each generator is in $(x_1,\ldots,x_n)$, which is precisely $z$. But then it is clear that this prime ideal is contained in the one generated by $x_1,\ldots,x_n$, which is exactly $z$.
\end{proof}
