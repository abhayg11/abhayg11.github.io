\mtexe{3.3.17}
\begin{proof}
	It is clear that a finite morphism is of finite type. Indeed, we can choose an open affine cover of the base, each preimage can be covered by a single affine scheme by assumption, and each corresponding ring map is finite, so certainly of finite-type (a module-generating set is certainly an algebra-generating set).
	
	To further show that a finite morphism is quasi-finite, it suffices to show it has finite fibers. But this is well-known for finite maps. Namely, if $f : X \to Y$ is finite and $y \in Y$, then we can choose an open affine $\Spec A$ of $y$ and know that $f^{-1}(V) = \Spec B$ is also affine. Then the fiber over $y$ is the set of primes of $B$ lying over $A$, of which there are finitely many. \\
	
	First, we show that closed morphisms are finite. Let $f : X \to Y$ be a closed morphism, and let $V \subseteq Y$ be an affine open. We'd like to compute $f^{-1}(V)$, and we know that we can do this via fiber products. In other words, the following diagram is Cartesian:
	\[ \begin{tikzcd} f^{-1}(V) \arrow[r] \arrow[d] & V \arrow[d] \\ X \arrow[r] & Y \end{tikzcd} \]
	Closed immersions are stable under base change, so we conclude that $f^{-1}(V) \to V$ is a closed immersion. But then $f^{-1}(V)$ is a closed subscheme of an affine scheme, so it is itself affine, given by $\Spec$ of the quotient ring. Furthermore, we know that if $A$ is a ring, then $A/I$ is a finite $A$-module, generated by $1$, so $f^{-1}(V) \to V$ comes from a finite ring map as required.
	
	Second, if $X \to Y$ and $Y \to Z$ are finite morphisms, then we'd like to show the composition is finite. But if $W \subseteq Z$ is affine, then the preimage in $Y$ is affine, and so its preimage in $X$ is affine. Further, the composition of finite ring maps is finite, so the corresponding ring maps being finite shows that the overall composition is finite, as desired.
	
	Finally, we'd like to show that finite morphisms are stable under base change. Let $f : X \to Y$ be a finite morphism and $g : Y' \to Y$ arbitrary. Write $X' = X \times_Y Y'$ with projections $\pi_1,\pi_2$, so that we seek to show that $\pi_2$ is finite. Note that we may assume that $Y$ is affine. Indeed, if we choose an open affine cover $\scU$ of $Y$, then their preimages under $g$ form an open cover of $Y'$. By exercise 3.3.15, if we can show that the restriction of $\pi_2$ to $\pi_2^{-1}(g^{-1}(U)) \to g^{-1}(U)$ is finite for each $U \in \scU$, then we will conclude that $\pi_2$ is itself finite. But this is exactly the base change of $f^{-1}(U) \to U$, which is itself finite. So, we will now assume $Y$ affine. Then, by assumption, $X$ is also affine.
	
	Let $U \subseteq Y'$ be an open affine. We'll compute the preimage $\pi_2^{-1}(U)$ via fiber products, giving the Cartesian diagram:
	\[ \begin{tikzcd} \pi_2^{-1}(U) \arrow[r] \arrow[d] & X' \arrow[r,"\pi_1"] \arrow[d,"\pi_2"] & X \arrow[d,"f"] \\ U \arrow[r] & Y' \arrow[r,"g"] & Y \end{tikzcd} \]
	But the fact that this is Cartesian shows that $\pi_2^{-1}(U) \cong X \times_Y U$, which is the product of affines over an affine scheme, so is affine. Further, more explicitly, if $Y = \Spec A$, $X = \Spec B$, and $U = \Spec C$, then $\pi_2^{-1}(U) \cong \Spec(B \otimes_A C)$, and $B \otimes_A C$ is finite over $C$ as desired. \\
	
	Now, let $\rho : A \to B$ be a finite ring homomorphism. If $I$ is an ideal of $B$, then the kernel of the composition $A \xrightarrow{\rho} B \to B/I$ is $\rho^{-1}(I)$, so $\rho$ induces an injection $A/\rho^{-1}(I) \to B/I$. Further, this map is finite since the images of the generators of $B$ over $A$ also work here. But a finite injection is surjective on $\Spec$, so $(\Spec\rho)(\Spec(B/I)) = \Spec(A/\rho^{-1}(I))$, i.e. $(\Spec\rho)(V(I)) = V(\rho^{-1}(I))$ as claimed. \\
	
	Finally, let's show that finite morphisms are proper. We've already seen that finite morphisms are of finite type. Second, to show that a finite morphism is separated, it suffices to show this locally on the base. But finite morphisms are affine, so are locally morphisms between affine schemes, and so are separated. Finally, we'd like to show that finite morphisms are universally closed. Since finite morphisms are stable under base change, it suffices to just show that they are closed. Finally, this property is again local on the base, so we'd like to show that if $f : X \to Y$ is finite with $Y$ affine, then $f$ is closed. But then $X$ is also affine, and so we'd like to show that if $\rho : A \to B$ is a ring homomorphism, then the image of a closed subset of $\Spec B$ under $\Spec \rho$ is closed. But this is exactly what we showed above: a closed subset of $\Spec B$ is of the form $V(I)$, and its image $V(\rho^{-1}(I))$ is closed.
\end{proof}
