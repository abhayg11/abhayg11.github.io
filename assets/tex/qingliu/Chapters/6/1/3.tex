\mtexe{6.1.3}
\begin{proof}
	This follows from the parallel fact on rings. Namely, for $\beta \in \widehat{B}$, there is some $b \in B$ with $\beta - b \in I\widehat{B}$, so that $d\beta \in db + d\left(I\widehat{B}\right) \subseteq \rho(\Omega^1_{B/A})+I\Omega^1_{\widehat{B}/A}$. This same decomposition shows that the left-hand side here is also a $\widehat{B}$-module (despite the first summand being only a $B$-module), and since we've just shown it contains a collection of $\widehat{B}$-module generators of the right hand side, they must be equal.
	
	The content of the next claim is that the final map is surjective. Indeed, it is obvious that the map $\rho : \Omega^1_{B/A} \to \Omega^1_{\widehat{B}/A}$ induces a map $\Omega^1_{B/A}/I^n \to \Omega^1_{\widehat{B}/A}/I^n$ and we can name its kernel $K_n$. Further, if $n \geq 1$, then any element of the latter module is represented by an element $\beta = b+u$ with $b \in \rho(\Omega^1_{B/A})$ and $u \in I\Omega^1_{\widehat{B}/A}$. But then the coset of $b$ maps to $\beta$ under the induced map $\rho$ above.
	
	Next, the commutativity of
	\[ \begin{tikzcd} \Omega^1_{B/A}/I^{n+1} \arrow[r] \arrow[d] & \Omega^1_{\widehat{B}/A}/I^{n+1} \arrow[d] \\ \Omega^1_{B/A}/I^n \arrow[r] & \Omega^1_{\widehat{B}/A}/I^n \end{tikzcd} \]
	implies the existence of the map $K_{n+1} \to K_n$. For surjectivity, suppose that $\omega+I^n\Omega^1_{B/A} \in K_n$. Then $\rho(\omega) \in I^n\Omega^1_{\widehat{B}/A}$, MISSING SOMETHING EASY PROBABLY?
\end{proof}
