\mtexe{6.1.2}
\begin{proof}
	There are potentially two different claims being made here. Either i) for any ring $A$, $\Omega^1_{A \llbracket T \rrbracket/A}$ is not a finitely-generated $A \llbracket T \rrbracket$-module, or ii) $\Omega^1_{A \llbracket T \rrbracket/A}$ is not necessarily a finitely-generated $A \llbracket T \rrbracket$-module. The former statement is false while the latter is true; we will exemplify this with two examples, one where the differentials are module-finite and one where it is not.
	
	First, let $A = \FF_2$ be the finite field with two elements. Then I claim $\Omega^1_{A \llbracket T \rrbracket/A}$ is generated by $dT$. It suffices to show that $df \in dTA \llbracket T \rrbracket/A$ for any $f \in A \llbracket T \rrbracket/A$. Write:
	\[ f = \sum_{i=0}^\infty a_iT^i = \sum_{i=0}^\infty a_{2i}T^{2i} + T\sum_{i=0}^\infty a_{2i+1}T^{2i} = \left(\sum_{i=0}^\infty a_{2i}T\right)^2 + T\left(\sum_{i=0}^\infty a_{2i+1}T\right)^2 \]
	i.e. $f = u^2 + Tv^2$ for some $u,v$. But then
	\[ df = d(u^2+Tv^2) = 2udu + 2Tvdv + v^2dt = v^2dt \]
	since $2=0$ here. More generally, we can show a similar example for any perfect field of positive characteristic. \\
	
	On the other hand, let $A = \QQ$, $B = \QQ \llbracket T \rrbracket$, $K = \Frac(B) = \QQ \llparenthesis T \rrparenthesis$, and $E$ a subextension of $K/A$ such that $E/A$ is transcendental and $K/E$ is algebraic (i.e. choose a transcendence basis). Since $K/E$ is algebraic and separable, $\Omega^1_{K/E} = 0$, and so $\Omega^1_{K/A} = K \otimes_E \Omega^1_{E/A}$. But also $\Omega^1_{K/A} = K \otimes_B \Omega^1_{B/A}$ since $B \to K$ is a localization. So, if $\Omega^1_{B/A}$ were a finite $B$-module, then $\Omega^1_{K/A} = K \otimes_E \Omega^1_{E/A}$ is a finite $K$-module, but $E/A$ is a purely transcendental field extension, so the dimension of the module of differentials is at least the transcendence degree. So it remains to show that $K/A$ doesn't have finite transcendence degree.
	
	This can be done with a counting argument: note that a field extension of $A = \QQ$ of finite transcendence degree is countable. Indeed, adjoining a purely transcendental element to a countable field yields a countable field, and any algebraic extension of a countable field is countable. On the other hand, $K$ is not countable, since it contains, for each subset $S \subseteq \NN$,
	\[ \sum_{i \in S} T^i \]
	So, $K/A$ cannot have finite transcendence degree. \\
	
	Now let us demonstrate the exact sequence. The first map is an inclusion of a submodule so obviously injective. Since $\varphi(dT) = dT$, it is clear that $\varphi$ is surjective. So we need to show exactness in the center. Suppose $\omega \in \bigcap_{n \geq 1} T^n\Omega_{A \llbracket t \rrbracket/A}$. Then for each $n \geq 1$ we can write $\omega = T^n\omega_n$ and $\varphi(\omega) = T^n\varphi(\omega_n)$. But each element of the image of $\varphi$ can be written in the form $fdT$ for $f \in A \llbracket t \rrbracket$, and we've just shown that $f$ is a multiple of $T^n$ for arbitrary $n$. Thus $f=0$ and we've shown that the image of the first map is contained in the kernel of $\varphi$.
	
	For the reverse inclusion, first consider elements of $\Omega_{A\llbracket t \rrbracket/A}$. By construction, any such element can be written as a finite sum of terms of the form $fdg$ with $f,g$ power series. Fix an $n \geq 1$, $g = a+T^nb$ with $a$ a polynomial of degree less than $n$. We then have
	\[ dg = d(a+T^nb) = da+nT^{n-1}bdT+T^ndb = a'dT+nT^{n-1}bdT+T^ndb = (g' - T^nb')dT + T^ndb = g'dT + T^n(b'dT-db) \]
	where we used the fact that $da = a'dT$ since $a$ only has finitely many summands. Now, note that
	\[ \varphi(fdg) = fg'dT = fdg - T^n(b'dT-db) \]
	Or in other words, $fdg - \varphi(fdg) \in T^n\Omega_{A\llbracket t \rrbracket/A}$ for any $f,g$. By summing over all the terms $fdg$ in an arbitrary differential, we conclude that $\omega-\varphi(\omega) \in T^n\Omega_{A\llbracket t \rrbracket/A}$ for any $\omega$. But then if $\omega \in \ker\varphi$ we get that $\omega \in T^n\Omega_{A\llbracket t \rrbracket/A}$, and since this is true for any $n$, we are done.
\end{proof}
