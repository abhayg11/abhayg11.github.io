\mtexe{6.1.2}
\begin{proof}
	There are potentially two different claims being made here. Either i) for any ring $A$, $\Omega^1_{A \llbracket t \rrbracket/A}$ is not a finitely-generated $A \llbracket t \rrbracket$-module, or ii) $\Omega^1_{A \llbracket t \rrbracket/A}$ is not necessarily a finitely-generated $A \llbracket t \rrbracket$-module. The former statement is false while the latter is true; we will exemplify this with two examples, one where the differentials are module-finite and one where it is not.
	
	First, let $A = \FF_2$ be the finite field with two elements. Then I claim $\Omega^1_{A \llbracket t \rrbracket/A}$ is generated by $dt$. It suffices to show that $df \in dtA \llbracket t \rrbracket/A$ for any $f \in A \llbracket t \rrbracket/A$. Write:
	\[ f = \sum_{i=0}^\infty a_it^i = \sum_{i=0}^\infty a_{2i}t^{2i} + t\sum_{i=0}^\infty a_{2i+1}t^{2i} = \left(\sum_{i=0}^\infty a_{2i}t\right)^2 + t\left(\sum_{i=0}^\infty a_{2i+1}t\right)^2 \]
	i.e. $f = u^2 + tv^2$ for some $u,v$. But then
	\[ df = d(u^2+tv^2) = 2udu + 2tvdv + v^2dt = v^2dt \]
	since $2=0$ here. More generally, we can show a similar example for any perfect field of positive characteristic. \\
	
	On the other hand, let $A = \QQ$, $B = \QQ \llbracket t \rrbracket$, $K = \Frac(B) = \QQ \llparenthesis t \rrparenthesis$, and $E$ a subextension of $K/A$ such that $E/A$ is transcendental and $K/E$ is algebraic (i.e. choose a transcendence basis). Since $K/E$ is algebraic and separable, $\Omega^1_{K/E} = 0$, and so $\Omega^1_{K/A} = K \otimes_E \Omega^1_{E/A}$. But also $\Omega^1_{K/A} = K \otimes_B \Omega^1_{B/A}$ since $B \to K$ is a localization. So, if $\Omega^1_{B/A}$ were a finite $B$-module, then $\Omega^1_{K/A} = K \otimes_E \Omega^1_{E/A}$ is a finite $K$-module, but $E/A$ is a purely transcendental field extension, so the dimension of the module of differentials is at least the transcendence degree. So it remains to show that $K/A$ doesn't have finite transcendence degree.
	
	This can be done with a counting argument: note that a field extension of $A = \QQ$ of finite transcendence degree is countable. Indeed, adjoining a purely transcendental element to a countable field yields a countable field, and any algebraic extension of a countable field is countable. On the other hand, $K$ is not countable, since it contains, for each subset $S \subseteq \NN$,
	\[ \sum_{i \in S} t^i \]
	So, $K/A$ cannot have finite transcendence degree.
\end{proof}
