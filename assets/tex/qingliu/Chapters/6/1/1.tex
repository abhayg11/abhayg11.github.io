\mtexe{6.1.1}
\begin{proof}
	First, note that $d$ is $A$-linear, for if $a \in A$ and $b,b' \in B$, then
	\begin{align*}
		d(ab) &= (ab) \otimes 1 - 1 \otimes (ab) = a(b \otimes 1 - 1 \otimes b) = adb \\
		d(b+b') &= (b+b') \otimes 1 - 1 \otimes (b+b') = (b \otimes 1 - 1 \otimes b)+(b' \otimes 1 - 1 \otimes b') = db+db'
	\end{align*}
	Further, it satisfies the product rule:
	\begin{align*}
	d(bb')-bdb'-b'db
		&= (bb') \otimes 1 - 1 \otimes (bb') - b(b' \otimes 1 - 1 \otimes b') - b'(b \otimes 1 - 1 \otimes b) \\
		&= (bb') \otimes 1 - 1 \otimes (bb') - b' \otimes b + 1 \otimes (bb') - b \otimes b' + 1 \otimes (bb') \\
		&= (bb') \otimes 1 - b' \otimes b - b \otimes b' + 1 \otimes (bb') \\
		&= (b \otimes 1 - 1 \otimes b)(b' \otimes 1 - 1 \otimes b')
	\end{align*}
	which is clearly in $I^2$, so is zero in $I/I^2$. Finally, elements of $A$ are constants since for $a \in A$:
	\[ da = a \otimes 1 - 1 \otimes a = a(1 \otimes 1 - 1 \otimes 1) = 0 \]
	as desired. \\
	
	Let $(\Omega_{B/A}^1, d')$ denote the actual module of relative differentials. Since we've just shown that $d$ is an $A$-derivation, we get a map $\varphi : \Omega_{B/A}^1 \to I/I^2$ of $B$-modules with $d = \varphi \circ d'$. To show this is an isomorphism, we'll construct an inverse. Note that the multiplication map $B \otimes_A B \to B$ is surjective, so $(B \otimes_A B)/I \cong B$. Hence, from proposition 6.1.8d, we get an exact sequence
	\[ I/I^2 \xrightarrow{\delta} \Omega^1_{(B \otimes_A B)/B} \otimes_{B \otimes_A B} B \xrightarrow{\alpha} \Omega^1_{B/B} \to 0 \]
	Let us investigate the second term of this sequence. First, note that $B \otimes_A B$ considered as a $B$-algebra is the base change of $B$ along the morphism $A \to B$. Since differentials commute with base change, we have
	\[ \Omega^1_{(B \otimes_A B)/B} \otimes_{B \otimes_A B} B \cong (\Omega^1_{B/A} \otimes_B (B \otimes_A B)) \otimes_{B \otimes_A B} B \]
	But now, this expression is given by taking the $B$-module $\Omega^1_{B/A}$ and base-changing along the composition $B \to B \otimes_A B \to B$, which is the identity. So, this is just $\Omega^1_{B/A}$ itself. After making these identifications, we have that $\delta$ is a map $I/I^2 \to \Omega^1_{B/A}$. I claim that $\delta$ is the two-sided inverse of $\varphi$ above, which will complete the proof.
	
	Suppose $d' : B \to M$ is an $A$-derivation of $B$ into $M$. Consider the function $B \times B \to M$ given by $(x,y) \mapsto yd'(x)$. This is $A$-bilinear since $d'$ is $A$-linear, so defines a map $\varphi : B \otimes_A B \to M$, which we can further restrict to $I$ to give a $A$-module map $\varphi : I \to M$. In fact, it is a $B$-module map since
	\[ \varphi(b(x \otimes y)) = \varphi(x \otimes (by)) = byd'(x) = b\varphi(x \otimes y) \]
	Further, this map is trivial on $I^2$. Indeed, suppose we have two elements of $I$, written in the form
	\[ \sum_i r_i \otimes s_i \text{ and } \sum_j x_j \otimes y_j \]
	for $r_i,s_i,x_j,y_j \in B$ and the sums taking place over some finite index set. The fact that they are in $I$ can be written as
	\[ \sum_i r_is_i = 0 = \sum_j x_jy_j \]
	Then, we can compute $\varphi$ of the product as:
	\begin{align*}
	\varphi\left(\left(\sum_i r_i \otimes s_i\right)\left(\sum_j x_j \otimes y_j\right)\right)
		&= \sum_{i,j} \varphi((r_ix_j) \otimes (s_iy_j)) \\
		&= \sum_{i,j} s_iy_jd'(r_ix_j) \\
		&= \sum_{i,j} (s_iy_jr_id'x_j + s_iy_jx_jd'r_i) \\
		&= 0
	\end{align*}
	since the first summand, summed over all $i$, gives 0; and the second summand, summed over all $j$, gives 0. So, $\varphi$ factors as a map $\varphi : I/I^2 \to M$.
	
	This is the morphism claimed by the universal property. Indeed, for $b \in B$, we have
	\[ \varphi(d(b)) = \varphi(b \otimes 1 - 1 \otimes b) = 1d'(b) - bd'(1) = d'(b) \]
	so that $\varphi \circ d = d'$. Finally, to show that $\varphi$ is unique, suppose that $\psi \circ d = d'$ for a $B$-module homomorphism $\psi : I/I^2 \to M$, and choose (a lift of) an element $x = \sum_i r_i \otimes s_i \in I/I^2$. Then
	\begin{align*}
	\psi(x)
	
	STILL DON'T KNOW HOW TO FINISH THE PROOF, BUT NOTE: $B$ acts on $I/I^2$ in the way defined above, but this is equivalent to acting on the left factor.
\end{proof}
