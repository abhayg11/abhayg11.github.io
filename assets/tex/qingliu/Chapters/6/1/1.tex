\mtexe{6.1.1}
\begin{proof}
	First, note that $d$ is $A$-linear, for if $a \in A$ and $b,b' \in B$, then
	\begin{align*}
		d(ab) &= (ab) \otimes 1 - 1 \otimes (ab) = a(b \otimes 1 - 1 \otimes b) = adb \\
		d(b+b') &= (b+b') \otimes 1 - 1 \otimes (b+b') = (b \otimes 1 - 1 \otimes b)+(b' \otimes 1 - 1 \otimes b') = db+db'
	\end{align*}
	Further, it satisfies the product rule:
	\begin{align*}
	d(bb')-bdb'-b'db
		&= (bb') \otimes 1 - 1 \otimes (bb') - b(b' \otimes 1 - 1 \otimes b') - b'(b \otimes 1 - 1 \otimes b) \\
		&= (bb') \otimes 1 - 1 \otimes (bb') - b' \otimes b + 1 \otimes (bb') - b \otimes b' + 1 \otimes (bb') \\
		&= (bb') \otimes 1 - b' \otimes b - b \otimes b' + 1 \otimes (bb') \\
		&= (b \otimes 1 - 1 \otimes b)(b' \otimes 1 - 1 \otimes b')
	\end{align*}
	which is clearly in $I^2$, so is zero in $I/I^2$. Finally, elements of $A$ are constants since for $a \in A$:
	\[ da = a \otimes 1 - 1 \otimes a = a(1 \otimes 1 - 1 \otimes 1) = 0 \]
	as desired. \\
	
	Let $(\Omega_{B/A}^1, d')$ denote the actual module of relative differentials. Since we've just shown that $d$ is an $A$-derivation, we get a map $\varphi : \Omega_{B/A}^1 \to I/I^2$ of $B$-modules with $d = \varphi \circ d'$. To show this is an isomorphism, we'll construct an inverse. Note that the multiplication map $B \otimes_A B \to B$ is surjective, so $(B \otimes_A B)/I \cong B$. Hence, from proposition 6.1.8d, we get an exact sequence
	\[ I/I^2 \xrightarrow{\delta} \Omega^1_{(B \otimes_A B)/B} \otimes_{B \otimes_A B} B \xrightarrow{\alpha} \Omega^1_{B/B} \to 0 \]
	Let us investigate the second term of this sequence. First, note that $B \otimes_A B$ considered as a $B$-algebra is the base change of $B$ along the morphism $A \to B$. Since differentials commute with base change, we have
	\[ \Omega^1_{(B \otimes_A B)/B} \otimes_{B \otimes_A B} B \cong (\Omega^1_{B/A} \otimes_B (B \otimes_A B)) \otimes_{B \otimes_A B} B \]
	But now, this expression is given by taking the $B$-module $\Omega^1_{B/A}$ and base-changing along the composition $B \to B \otimes_A B \to B$, which is the identity. So, this is just $\Omega^1_{B/A}$ itself. After making these identifications, we have that $\delta$ is a map $I/I^2 \to \Omega^1_{B/A}$. I claim that $\delta$ is the two-sided inverse of $\varphi$ above, which will complete the proof. Unwrapping the above identifications gives
	\[ \delta(\sum_i b_i \otimes c_i) = \sum_i c_id'b_i \]
	Applying $\varphi$ to this gives
	\[ \varphi(\delta(\sum_i b_i \otimes c_i)) = \sum_i c_i(b_i \otimes 1 - 1 \otimes b_i) = \sum_i b_i \otimes c_i - 1 \otimes (\sum_i b_ic_i) = \sum_i b_i \otimes c_i \]
	since $\sum_i b_ic_i = 0$. So, this composition is the identity as desired. Conversely, $\Omega_{B/A}^1$ is generated as a $B$-module by $d'b$ for $b \in B$. On one such element, we have
	\[ \delta(\varphi(d'b)) = \delta(db) = \delta(b \otimes 1 - 1 \otimes b) = 1d'b - bd'1 = d'b \]
	since $d'$ is an $A$-derivation and $1 \in A$. This completes the proof.
\end{proof}
