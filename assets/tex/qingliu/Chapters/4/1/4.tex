\mtexe{4.1.4}
\begin{proof}
	If $X$ is globally reduced, integral, or normal, then certainly this property holds locally at all of the closed points, since it holds at all of the points. Conversely, suppose that $X$ has property $\mcP$ at all of the closed points. Note that varieties are quasi-compact, so if $x \in X$ is arbitrary, then $\overline{\{x\}}$ contains some closed point $y$. Then choose an open affine neighborhood $U$ of $y$, note that $x \in U$ by assumption, and note that $\scO_{X,x} = \scO_{U,x}$ is a localization of $\scO_{U,y} = \scO_{X,y}$. Since the latter has $\mcP$, the former does, too. So, $X$ has $\mcP$ at all points.
	
	For reducedness, the argument is complete. For normality and integrality, we'd like to further show that $X$ is irreducible. If not, since $X$ is connected, it has two distinct irreducible components that intersect nontrivially. But if $x$ is in this intersection, and $U$ is an affine neighborhood of $x$, then $\scO_{X,x}$ is a domain, but it contains at least two minimal primes corresponding to the generic points of the two components intersecting in $x$, both of which must be contained in $U$, and so contained as prime ideals in $x$, and so surviving in the localization $\scO_{X,x}$. This is a contradiction since a domain has a unique minimal prime, so we indeed have that $X$ is irreducible in these cases.
\end{proof}
