\mtexe{4.1.9}
\begin{proof}
	First, I'd like to clarify why we may assume $U$ is the complement of a closed point. Note that since $X = \Spec A$ is affine, a closed subset is of the form $V(I)$, but since $A$ is Dedekind, $V(I)$ is precisely the set of primes dividing $I$, which is finite. So, if we can show that the complement of a closed point is affine, then inductively we have that the complement of any proper closed subset is affine. \\
	
	Note that $\frm_i \not\subseteq \frm_0$ and $\frm_i \not\subseteq \frm_i^2$. By prime avoidance, we may choose $t_i \in \frm_i \setminus (\frm_0 \cup \frm_i^2)$. I claim that $t_iA_{\frm_i} = \frm_iA_{\frm_i}$. Indeed, it is immediate since $A_{\frm_i}$ is a DVR and the image of $t_i$ is contained in the maximal ideal but not its square. \\
	
	As suggested, let $f = t^{-1}\prod_{i=1}^n t_i^{d_i} \in \Frac(A)$ for yet-to-be-determined exponents $d_i$. Since $t_i \notin \frm_0$, the $\frm_0$-adic valuation is $-1+0+\cdots+0 = -1$, and so this is certainly not in $A$. The expression also makes it clear that $f \in A_t = \scO_X(D(t))$. In order to argue that it actually extends to $\scO_X(U)$, we need to see that it locally defines a function at each of $x_1,\ldots,x_n$. In other words, we are looking for the $\frm_i$-adic valuation of $f$ for each $i$. But this is nonnegative as long as $d_i \geq \nu_{\frm_i}(t)$, for example. \\
	
	The prescription of the morphism is also largely the proof of its existence and uniqueness. More precisely, knowing the morphism on an open cover gives uniqueness, and for existence we only need to show that the proposed restrictions are compatible. But this is also clear from the definition since it is defined globally by $T_1/T_0 \mapsto f$. \\
	
	Let $p : \PP^1_X \to X$ denote the canonical map, and note that $p \circ \varphi : X \to X$ is a morphism of $A$-schemes, in that it is compatible with the structure maps $X \to \Spec A$. But of course $X = \Spec A$ and these structure maps are the identity, so $p \circ \varphi = \id_X$, so that $\varphi$ is a section of $p$. Then $\varphi(U) \subseteq D_{+}(T_0)$, so $U \subseteq \varphi^{-1}(D_{+}(T_0))$, but $\varphi(x_0) \notin D_{+}(T_0)$ since on the restriction to $V$, we have $T_0/T_1 \mapsto 1/f \in \frm_0A[f^{-1}]$. But then $p$ is projective, so separated, and so $\varphi$ is a closed immersion, whence the restriction $\varphi|_U$ is a closed immersion to the affine scheme $D_{+}(T_0)$. So $U$ is affine, given by the corresponding quotient ring, which is isomorphic to the image of $A[T_1/T_0] \to \scO_X(U)$, $T_1/T_0 \mapsto f$; which is clearly $A[f]$, as desired.
\end{proof}
